\documentclass[11pt]{article}

    \usepackage[breakable]{tcolorbox}
    \usepackage{parskip} % Stop auto-indenting (to mimic markdown behaviour)
    
    \usepackage{iftex}
    \ifPDFTeX
    	\usepackage[T1]{fontenc}
    	\usepackage{mathpazo}
    \else
    	\usepackage{fontspec}
    \fi

    % Basic figure setup, for now with no caption control since it's done
    % automatically by Pandoc (which extracts ![](path) syntax from Markdown).
    \usepackage{graphicx}
    % Maintain compatibility with old templates. Remove in nbconvert 6.0
    \let\Oldincludegraphics\includegraphics
    % Ensure that by default, figures have no caption (until we provide a
    % proper Figure object with a Caption API and a way to capture that
    % in the conversion process - todo).
    \usepackage{caption}
    \DeclareCaptionFormat{nocaption}{}
    \captionsetup{format=nocaption,aboveskip=0pt,belowskip=0pt}

    \usepackage{float}
    \floatplacement{figure}{H} % forces figures to be placed at the correct location
    \usepackage{xcolor} % Allow colors to be defined
    \usepackage{enumerate} % Needed for markdown enumerations to work
    \usepackage{geometry} % Used to adjust the document margins
    \usepackage{amsmath} % Equations
    \usepackage{amssymb} % Equations
    \usepackage{textcomp} % defines textquotesingle
    % Hack from http://tex.stackexchange.com/a/47451/13684:
    \AtBeginDocument{%
        \def\PYZsq{\textquotesingle}% Upright quotes in Pygmentized code
    }
    \usepackage{upquote} % Upright quotes for verbatim code
    \usepackage{eurosym} % defines \euro
    \usepackage[mathletters]{ucs} % Extended unicode (utf-8) support
    \usepackage{fancyvrb} % verbatim replacement that allows latex
    \usepackage{grffile} % extends the file name processing of package graphics 
                         % to support a larger range
    \makeatletter % fix for old versions of grffile with XeLaTeX
    \@ifpackagelater{grffile}{2019/11/01}
    {
      % Do nothing on new versions
    }
    {
      \def\Gread@@xetex#1{%
        \IfFileExists{"\Gin@base".bb}%
        {\Gread@eps{\Gin@base.bb}}%
        {\Gread@@xetex@aux#1}%
      }
    }
    \makeatother
    \usepackage[Export]{adjustbox} % Used to constrain images to a maximum size
    \adjustboxset{max size={0.9\linewidth}{0.9\paperheight}}

    % The hyperref package gives us a pdf with properly built
    % internal navigation ('pdf bookmarks' for the table of contents,
    % internal cross-reference links, web links for URLs, etc.)
    \usepackage{hyperref}
    % The default LaTeX title has an obnoxious amount of whitespace. By default,
    % titling removes some of it. It also provides customization options.
    \usepackage{titling}
    \usepackage{longtable} % longtable support required by pandoc >1.10
    \usepackage{booktabs}  % table support for pandoc > 1.12.2
    \usepackage[inline]{enumitem} % IRkernel/repr support (it uses the enumerate* environment)
    \usepackage[normalem]{ulem} % ulem is needed to support strikethroughs (\sout)
                                % normalem makes italics be italics, not underlines
    \usepackage{mathrsfs}
    

    
    % Colors for the hyperref package
    \definecolor{urlcolor}{rgb}{0,.145,.698}
    \definecolor{linkcolor}{rgb}{.71,0.21,0.01}
    \definecolor{citecolor}{rgb}{.12,.54,.11}

    % ANSI colors
    \definecolor{ansi-black}{HTML}{3E424D}
    \definecolor{ansi-black-intense}{HTML}{282C36}
    \definecolor{ansi-red}{HTML}{E75C58}
    \definecolor{ansi-red-intense}{HTML}{B22B31}
    \definecolor{ansi-green}{HTML}{00A250}
    \definecolor{ansi-green-intense}{HTML}{007427}
    \definecolor{ansi-yellow}{HTML}{DDB62B}
    \definecolor{ansi-yellow-intense}{HTML}{B27D12}
    \definecolor{ansi-blue}{HTML}{208FFB}
    \definecolor{ansi-blue-intense}{HTML}{0065CA}
    \definecolor{ansi-magenta}{HTML}{D160C4}
    \definecolor{ansi-magenta-intense}{HTML}{A03196}
    \definecolor{ansi-cyan}{HTML}{60C6C8}
    \definecolor{ansi-cyan-intense}{HTML}{258F8F}
    \definecolor{ansi-white}{HTML}{C5C1B4}
    \definecolor{ansi-white-intense}{HTML}{A1A6B2}
    \definecolor{ansi-default-inverse-fg}{HTML}{FFFFFF}
    \definecolor{ansi-default-inverse-bg}{HTML}{000000}

    % common color for the border for error outputs.
    \definecolor{outerrorbackground}{HTML}{FFDFDF}

    % commands and environments needed by pandoc snippets
    % extracted from the output of `pandoc -s`
    \providecommand{\tightlist}{%
      \setlength{\itemsep}{0pt}\setlength{\parskip}{0pt}}
    \DefineVerbatimEnvironment{Highlighting}{Verbatim}{commandchars=\\\{\}}
    % Add ',fontsize=\small' for more characters per line
    \newenvironment{Shaded}{}{}
    \newcommand{\KeywordTok}[1]{\textcolor[rgb]{0.00,0.44,0.13}{\textbf{{#1}}}}
    \newcommand{\DataTypeTok}[1]{\textcolor[rgb]{0.56,0.13,0.00}{{#1}}}
    \newcommand{\DecValTok}[1]{\textcolor[rgb]{0.25,0.63,0.44}{{#1}}}
    \newcommand{\BaseNTok}[1]{\textcolor[rgb]{0.25,0.63,0.44}{{#1}}}
    \newcommand{\FloatTok}[1]{\textcolor[rgb]{0.25,0.63,0.44}{{#1}}}
    \newcommand{\CharTok}[1]{\textcolor[rgb]{0.25,0.44,0.63}{{#1}}}
    \newcommand{\StringTok}[1]{\textcolor[rgb]{0.25,0.44,0.63}{{#1}}}
    \newcommand{\CommentTok}[1]{\textcolor[rgb]{0.38,0.63,0.69}{\textit{{#1}}}}
    \newcommand{\OtherTok}[1]{\textcolor[rgb]{0.00,0.44,0.13}{{#1}}}
    \newcommand{\AlertTok}[1]{\textcolor[rgb]{1.00,0.00,0.00}{\textbf{{#1}}}}
    \newcommand{\FunctionTok}[1]{\textcolor[rgb]{0.02,0.16,0.49}{{#1}}}
    \newcommand{\RegionMarkerTok}[1]{{#1}}
    \newcommand{\ErrorTok}[1]{\textcolor[rgb]{1.00,0.00,0.00}{\textbf{{#1}}}}
    \newcommand{\NormalTok}[1]{{#1}}
    
    % Additional commands for more recent versions of Pandoc
    \newcommand{\ConstantTok}[1]{\textcolor[rgb]{0.53,0.00,0.00}{{#1}}}
    \newcommand{\SpecialCharTok}[1]{\textcolor[rgb]{0.25,0.44,0.63}{{#1}}}
    \newcommand{\VerbatimStringTok}[1]{\textcolor[rgb]{0.25,0.44,0.63}{{#1}}}
    \newcommand{\SpecialStringTok}[1]{\textcolor[rgb]{0.73,0.40,0.53}{{#1}}}
    \newcommand{\ImportTok}[1]{{#1}}
    \newcommand{\DocumentationTok}[1]{\textcolor[rgb]{0.73,0.13,0.13}{\textit{{#1}}}}
    \newcommand{\AnnotationTok}[1]{\textcolor[rgb]{0.38,0.63,0.69}{\textbf{\textit{{#1}}}}}
    \newcommand{\CommentVarTok}[1]{\textcolor[rgb]{0.38,0.63,0.69}{\textbf{\textit{{#1}}}}}
    \newcommand{\VariableTok}[1]{\textcolor[rgb]{0.10,0.09,0.49}{{#1}}}
    \newcommand{\ControlFlowTok}[1]{\textcolor[rgb]{0.00,0.44,0.13}{\textbf{{#1}}}}
    \newcommand{\OperatorTok}[1]{\textcolor[rgb]{0.40,0.40,0.40}{{#1}}}
    \newcommand{\BuiltInTok}[1]{{#1}}
    \newcommand{\ExtensionTok}[1]{{#1}}
    \newcommand{\PreprocessorTok}[1]{\textcolor[rgb]{0.74,0.48,0.00}{{#1}}}
    \newcommand{\AttributeTok}[1]{\textcolor[rgb]{0.49,0.56,0.16}{{#1}}}
    \newcommand{\InformationTok}[1]{\textcolor[rgb]{0.38,0.63,0.69}{\textbf{\textit{{#1}}}}}
    \newcommand{\WarningTok}[1]{\textcolor[rgb]{0.38,0.63,0.69}{\textbf{\textit{{#1}}}}}
    
    
    % Define a nice break command that doesn't care if a line doesn't already
    % exist.
    \def\br{\hspace*{\fill} \\* }
    % Math Jax compatibility definitions
    \def\gt{>}
    \def\lt{<}
    \let\Oldtex\TeX
    \let\Oldlatex\LaTeX
    \renewcommand{\TeX}{\textrm{\Oldtex}}
    \renewcommand{\LaTeX}{\textrm{\Oldlatex}}
    % Document parameters
    % Document title
    \title{breast\_cancer\_pca}
    
    
    
    
    
% Pygments definitions
\makeatletter
\def\PY@reset{\let\PY@it=\relax \let\PY@bf=\relax%
    \let\PY@ul=\relax \let\PY@tc=\relax%
    \let\PY@bc=\relax \let\PY@ff=\relax}
\def\PY@tok#1{\csname PY@tok@#1\endcsname}
\def\PY@toks#1+{\ifx\relax#1\empty\else%
    \PY@tok{#1}\expandafter\PY@toks\fi}
\def\PY@do#1{\PY@bc{\PY@tc{\PY@ul{%
    \PY@it{\PY@bf{\PY@ff{#1}}}}}}}
\def\PY#1#2{\PY@reset\PY@toks#1+\relax+\PY@do{#2}}

\@namedef{PY@tok@w}{\def\PY@tc##1{\textcolor[rgb]{0.73,0.73,0.73}{##1}}}
\@namedef{PY@tok@c}{\let\PY@it=\textit\def\PY@tc##1{\textcolor[rgb]{0.25,0.50,0.50}{##1}}}
\@namedef{PY@tok@cp}{\def\PY@tc##1{\textcolor[rgb]{0.74,0.48,0.00}{##1}}}
\@namedef{PY@tok@k}{\let\PY@bf=\textbf\def\PY@tc##1{\textcolor[rgb]{0.00,0.50,0.00}{##1}}}
\@namedef{PY@tok@kp}{\def\PY@tc##1{\textcolor[rgb]{0.00,0.50,0.00}{##1}}}
\@namedef{PY@tok@kt}{\def\PY@tc##1{\textcolor[rgb]{0.69,0.00,0.25}{##1}}}
\@namedef{PY@tok@o}{\def\PY@tc##1{\textcolor[rgb]{0.40,0.40,0.40}{##1}}}
\@namedef{PY@tok@ow}{\let\PY@bf=\textbf\def\PY@tc##1{\textcolor[rgb]{0.67,0.13,1.00}{##1}}}
\@namedef{PY@tok@nb}{\def\PY@tc##1{\textcolor[rgb]{0.00,0.50,0.00}{##1}}}
\@namedef{PY@tok@nf}{\def\PY@tc##1{\textcolor[rgb]{0.00,0.00,1.00}{##1}}}
\@namedef{PY@tok@nc}{\let\PY@bf=\textbf\def\PY@tc##1{\textcolor[rgb]{0.00,0.00,1.00}{##1}}}
\@namedef{PY@tok@nn}{\let\PY@bf=\textbf\def\PY@tc##1{\textcolor[rgb]{0.00,0.00,1.00}{##1}}}
\@namedef{PY@tok@ne}{\let\PY@bf=\textbf\def\PY@tc##1{\textcolor[rgb]{0.82,0.25,0.23}{##1}}}
\@namedef{PY@tok@nv}{\def\PY@tc##1{\textcolor[rgb]{0.10,0.09,0.49}{##1}}}
\@namedef{PY@tok@no}{\def\PY@tc##1{\textcolor[rgb]{0.53,0.00,0.00}{##1}}}
\@namedef{PY@tok@nl}{\def\PY@tc##1{\textcolor[rgb]{0.63,0.63,0.00}{##1}}}
\@namedef{PY@tok@ni}{\let\PY@bf=\textbf\def\PY@tc##1{\textcolor[rgb]{0.60,0.60,0.60}{##1}}}
\@namedef{PY@tok@na}{\def\PY@tc##1{\textcolor[rgb]{0.49,0.56,0.16}{##1}}}
\@namedef{PY@tok@nt}{\let\PY@bf=\textbf\def\PY@tc##1{\textcolor[rgb]{0.00,0.50,0.00}{##1}}}
\@namedef{PY@tok@nd}{\def\PY@tc##1{\textcolor[rgb]{0.67,0.13,1.00}{##1}}}
\@namedef{PY@tok@s}{\def\PY@tc##1{\textcolor[rgb]{0.73,0.13,0.13}{##1}}}
\@namedef{PY@tok@sd}{\let\PY@it=\textit\def\PY@tc##1{\textcolor[rgb]{0.73,0.13,0.13}{##1}}}
\@namedef{PY@tok@si}{\let\PY@bf=\textbf\def\PY@tc##1{\textcolor[rgb]{0.73,0.40,0.53}{##1}}}
\@namedef{PY@tok@se}{\let\PY@bf=\textbf\def\PY@tc##1{\textcolor[rgb]{0.73,0.40,0.13}{##1}}}
\@namedef{PY@tok@sr}{\def\PY@tc##1{\textcolor[rgb]{0.73,0.40,0.53}{##1}}}
\@namedef{PY@tok@ss}{\def\PY@tc##1{\textcolor[rgb]{0.10,0.09,0.49}{##1}}}
\@namedef{PY@tok@sx}{\def\PY@tc##1{\textcolor[rgb]{0.00,0.50,0.00}{##1}}}
\@namedef{PY@tok@m}{\def\PY@tc##1{\textcolor[rgb]{0.40,0.40,0.40}{##1}}}
\@namedef{PY@tok@gh}{\let\PY@bf=\textbf\def\PY@tc##1{\textcolor[rgb]{0.00,0.00,0.50}{##1}}}
\@namedef{PY@tok@gu}{\let\PY@bf=\textbf\def\PY@tc##1{\textcolor[rgb]{0.50,0.00,0.50}{##1}}}
\@namedef{PY@tok@gd}{\def\PY@tc##1{\textcolor[rgb]{0.63,0.00,0.00}{##1}}}
\@namedef{PY@tok@gi}{\def\PY@tc##1{\textcolor[rgb]{0.00,0.63,0.00}{##1}}}
\@namedef{PY@tok@gr}{\def\PY@tc##1{\textcolor[rgb]{1.00,0.00,0.00}{##1}}}
\@namedef{PY@tok@ge}{\let\PY@it=\textit}
\@namedef{PY@tok@gs}{\let\PY@bf=\textbf}
\@namedef{PY@tok@gp}{\let\PY@bf=\textbf\def\PY@tc##1{\textcolor[rgb]{0.00,0.00,0.50}{##1}}}
\@namedef{PY@tok@go}{\def\PY@tc##1{\textcolor[rgb]{0.53,0.53,0.53}{##1}}}
\@namedef{PY@tok@gt}{\def\PY@tc##1{\textcolor[rgb]{0.00,0.27,0.87}{##1}}}
\@namedef{PY@tok@err}{\def\PY@bc##1{{\setlength{\fboxsep}{\string -\fboxrule}\fcolorbox[rgb]{1.00,0.00,0.00}{1,1,1}{\strut ##1}}}}
\@namedef{PY@tok@kc}{\let\PY@bf=\textbf\def\PY@tc##1{\textcolor[rgb]{0.00,0.50,0.00}{##1}}}
\@namedef{PY@tok@kd}{\let\PY@bf=\textbf\def\PY@tc##1{\textcolor[rgb]{0.00,0.50,0.00}{##1}}}
\@namedef{PY@tok@kn}{\let\PY@bf=\textbf\def\PY@tc##1{\textcolor[rgb]{0.00,0.50,0.00}{##1}}}
\@namedef{PY@tok@kr}{\let\PY@bf=\textbf\def\PY@tc##1{\textcolor[rgb]{0.00,0.50,0.00}{##1}}}
\@namedef{PY@tok@bp}{\def\PY@tc##1{\textcolor[rgb]{0.00,0.50,0.00}{##1}}}
\@namedef{PY@tok@fm}{\def\PY@tc##1{\textcolor[rgb]{0.00,0.00,1.00}{##1}}}
\@namedef{PY@tok@vc}{\def\PY@tc##1{\textcolor[rgb]{0.10,0.09,0.49}{##1}}}
\@namedef{PY@tok@vg}{\def\PY@tc##1{\textcolor[rgb]{0.10,0.09,0.49}{##1}}}
\@namedef{PY@tok@vi}{\def\PY@tc##1{\textcolor[rgb]{0.10,0.09,0.49}{##1}}}
\@namedef{PY@tok@vm}{\def\PY@tc##1{\textcolor[rgb]{0.10,0.09,0.49}{##1}}}
\@namedef{PY@tok@sa}{\def\PY@tc##1{\textcolor[rgb]{0.73,0.13,0.13}{##1}}}
\@namedef{PY@tok@sb}{\def\PY@tc##1{\textcolor[rgb]{0.73,0.13,0.13}{##1}}}
\@namedef{PY@tok@sc}{\def\PY@tc##1{\textcolor[rgb]{0.73,0.13,0.13}{##1}}}
\@namedef{PY@tok@dl}{\def\PY@tc##1{\textcolor[rgb]{0.73,0.13,0.13}{##1}}}
\@namedef{PY@tok@s2}{\def\PY@tc##1{\textcolor[rgb]{0.73,0.13,0.13}{##1}}}
\@namedef{PY@tok@sh}{\def\PY@tc##1{\textcolor[rgb]{0.73,0.13,0.13}{##1}}}
\@namedef{PY@tok@s1}{\def\PY@tc##1{\textcolor[rgb]{0.73,0.13,0.13}{##1}}}
\@namedef{PY@tok@mb}{\def\PY@tc##1{\textcolor[rgb]{0.40,0.40,0.40}{##1}}}
\@namedef{PY@tok@mf}{\def\PY@tc##1{\textcolor[rgb]{0.40,0.40,0.40}{##1}}}
\@namedef{PY@tok@mh}{\def\PY@tc##1{\textcolor[rgb]{0.40,0.40,0.40}{##1}}}
\@namedef{PY@tok@mi}{\def\PY@tc##1{\textcolor[rgb]{0.40,0.40,0.40}{##1}}}
\@namedef{PY@tok@il}{\def\PY@tc##1{\textcolor[rgb]{0.40,0.40,0.40}{##1}}}
\@namedef{PY@tok@mo}{\def\PY@tc##1{\textcolor[rgb]{0.40,0.40,0.40}{##1}}}
\@namedef{PY@tok@ch}{\let\PY@it=\textit\def\PY@tc##1{\textcolor[rgb]{0.25,0.50,0.50}{##1}}}
\@namedef{PY@tok@cm}{\let\PY@it=\textit\def\PY@tc##1{\textcolor[rgb]{0.25,0.50,0.50}{##1}}}
\@namedef{PY@tok@cpf}{\let\PY@it=\textit\def\PY@tc##1{\textcolor[rgb]{0.25,0.50,0.50}{##1}}}
\@namedef{PY@tok@c1}{\let\PY@it=\textit\def\PY@tc##1{\textcolor[rgb]{0.25,0.50,0.50}{##1}}}
\@namedef{PY@tok@cs}{\let\PY@it=\textit\def\PY@tc##1{\textcolor[rgb]{0.25,0.50,0.50}{##1}}}

\def\PYZbs{\char`\\}
\def\PYZus{\char`\_}
\def\PYZob{\char`\{}
\def\PYZcb{\char`\}}
\def\PYZca{\char`\^}
\def\PYZam{\char`\&}
\def\PYZlt{\char`\<}
\def\PYZgt{\char`\>}
\def\PYZsh{\char`\#}
\def\PYZpc{\char`\%}
\def\PYZdl{\char`\$}
\def\PYZhy{\char`\-}
\def\PYZsq{\char`\'}
\def\PYZdq{\char`\"}
\def\PYZti{\char`\~}
% for compatibility with earlier versions
\def\PYZat{@}
\def\PYZlb{[}
\def\PYZrb{]}
\makeatother


    % For linebreaks inside Verbatim environment from package fancyvrb. 
    \makeatletter
        \newbox\Wrappedcontinuationbox 
        \newbox\Wrappedvisiblespacebox 
        \newcommand*\Wrappedvisiblespace {\textcolor{red}{\textvisiblespace}} 
        \newcommand*\Wrappedcontinuationsymbol {\textcolor{red}{\llap{\tiny$\m@th\hookrightarrow$}}} 
        \newcommand*\Wrappedcontinuationindent {3ex } 
        \newcommand*\Wrappedafterbreak {\kern\Wrappedcontinuationindent\copy\Wrappedcontinuationbox} 
        % Take advantage of the already applied Pygments mark-up to insert 
        % potential linebreaks for TeX processing. 
        %        {, <, #, %, $, ' and ": go to next line. 
        %        _, }, ^, &, >, - and ~: stay at end of broken line. 
        % Use of \textquotesingle for straight quote. 
        \newcommand*\Wrappedbreaksatspecials {% 
            \def\PYGZus{\discretionary{\char`\_}{\Wrappedafterbreak}{\char`\_}}% 
            \def\PYGZob{\discretionary{}{\Wrappedafterbreak\char`\{}{\char`\{}}% 
            \def\PYGZcb{\discretionary{\char`\}}{\Wrappedafterbreak}{\char`\}}}% 
            \def\PYGZca{\discretionary{\char`\^}{\Wrappedafterbreak}{\char`\^}}% 
            \def\PYGZam{\discretionary{\char`\&}{\Wrappedafterbreak}{\char`\&}}% 
            \def\PYGZlt{\discretionary{}{\Wrappedafterbreak\char`\<}{\char`\<}}% 
            \def\PYGZgt{\discretionary{\char`\>}{\Wrappedafterbreak}{\char`\>}}% 
            \def\PYGZsh{\discretionary{}{\Wrappedafterbreak\char`\#}{\char`\#}}% 
            \def\PYGZpc{\discretionary{}{\Wrappedafterbreak\char`\%}{\char`\%}}% 
            \def\PYGZdl{\discretionary{}{\Wrappedafterbreak\char`\$}{\char`\$}}% 
            \def\PYGZhy{\discretionary{\char`\-}{\Wrappedafterbreak}{\char`\-}}% 
            \def\PYGZsq{\discretionary{}{\Wrappedafterbreak\textquotesingle}{\textquotesingle}}% 
            \def\PYGZdq{\discretionary{}{\Wrappedafterbreak\char`\"}{\char`\"}}% 
            \def\PYGZti{\discretionary{\char`\~}{\Wrappedafterbreak}{\char`\~}}% 
        } 
        % Some characters . , ; ? ! / are not pygmentized. 
        % This macro makes them "active" and they will insert potential linebreaks 
        \newcommand*\Wrappedbreaksatpunct {% 
            \lccode`\~`\.\lowercase{\def~}{\discretionary{\hbox{\char`\.}}{\Wrappedafterbreak}{\hbox{\char`\.}}}% 
            \lccode`\~`\,\lowercase{\def~}{\discretionary{\hbox{\char`\,}}{\Wrappedafterbreak}{\hbox{\char`\,}}}% 
            \lccode`\~`\;\lowercase{\def~}{\discretionary{\hbox{\char`\;}}{\Wrappedafterbreak}{\hbox{\char`\;}}}% 
            \lccode`\~`\:\lowercase{\def~}{\discretionary{\hbox{\char`\:}}{\Wrappedafterbreak}{\hbox{\char`\:}}}% 
            \lccode`\~`\?\lowercase{\def~}{\discretionary{\hbox{\char`\?}}{\Wrappedafterbreak}{\hbox{\char`\?}}}% 
            \lccode`\~`\!\lowercase{\def~}{\discretionary{\hbox{\char`\!}}{\Wrappedafterbreak}{\hbox{\char`\!}}}% 
            \lccode`\~`\/\lowercase{\def~}{\discretionary{\hbox{\char`\/}}{\Wrappedafterbreak}{\hbox{\char`\/}}}% 
            \catcode`\.\active
            \catcode`\,\active 
            \catcode`\;\active
            \catcode`\:\active
            \catcode`\?\active
            \catcode`\!\active
            \catcode`\/\active 
            \lccode`\~`\~ 	
        }
    \makeatother

    \let\OriginalVerbatim=\Verbatim
    \makeatletter
    \renewcommand{\Verbatim}[1][1]{%
        %\parskip\z@skip
        \sbox\Wrappedcontinuationbox {\Wrappedcontinuationsymbol}%
        \sbox\Wrappedvisiblespacebox {\FV@SetupFont\Wrappedvisiblespace}%
        \def\FancyVerbFormatLine ##1{\hsize\linewidth
            \vtop{\raggedright\hyphenpenalty\z@\exhyphenpenalty\z@
                \doublehyphendemerits\z@\finalhyphendemerits\z@
                \strut ##1\strut}%
        }%
        % If the linebreak is at a space, the latter will be displayed as visible
        % space at end of first line, and a continuation symbol starts next line.
        % Stretch/shrink are however usually zero for typewriter font.
        \def\FV@Space {%
            \nobreak\hskip\z@ plus\fontdimen3\font minus\fontdimen4\font
            \discretionary{\copy\Wrappedvisiblespacebox}{\Wrappedafterbreak}
            {\kern\fontdimen2\font}%
        }%
        
        % Allow breaks at special characters using \PYG... macros.
        \Wrappedbreaksatspecials
        % Breaks at punctuation characters . , ; ? ! and / need catcode=\active 	
        \OriginalVerbatim[#1,codes*=\Wrappedbreaksatpunct]%
    }
    \makeatother

    % Exact colors from NB
    \definecolor{incolor}{HTML}{303F9F}
    \definecolor{outcolor}{HTML}{D84315}
    \definecolor{cellborder}{HTML}{CFCFCF}
    \definecolor{cellbackground}{HTML}{F7F7F7}
    
    % prompt
    \makeatletter
    \newcommand{\boxspacing}{\kern\kvtcb@left@rule\kern\kvtcb@boxsep}
    \makeatother
    \newcommand{\prompt}[4]{
        {\ttfamily\llap{{\color{#2}[#3]:\hspace{3pt}#4}}\vspace{-\baselineskip}}
    }
    

    
    % Prevent overflowing lines due to hard-to-break entities
    \sloppy 
    % Setup hyperref package
    \hypersetup{
      breaklinks=true,  % so long urls are correctly broken across lines
      colorlinks=true,
      urlcolor=urlcolor,
      linkcolor=linkcolor,
      citecolor=citecolor,
      }
    % Slightly bigger margins than the latex defaults
    
    \geometry{verbose,tmargin=1in,bmargin=1in,lmargin=1in,rmargin=1in}
    
    

\begin{document}
    
    \maketitle
    
    

    
    \hypertarget{pca-uxacfcuxc81c}{%
\subsection{PCA 과제}\label{pca-uxacfcuxc81c}}

\hypertarget{sklearn.datasets.load_breast_canceruxc5d0-pcauxb97c-uxc801uxc6a9uxd558uxc5ec-2d-spaceuxb85c-uxb370uxc774uxd130-uxcc28uxc6d0-uxcd95uxc18c}{%
\subsubsection{1. sklearn.datasets.load\_breast\_cancer에 PCA를 적용하여
2D space로 데이터 차원
축소}\label{sklearn.datasets.load_breast_canceruxc5d0-pcauxb97c-uxc801uxc6a9uxd558uxc5ec-2d-spaceuxb85c-uxb370uxc774uxd130-uxcc28uxc6d0-uxcd95uxc18c}}

\hypertarget{pca-componentsuxc758-feature-importance-visualization}{%
\subsubsection{2. PCA components의 feature importance
visualization}\label{pca-componentsuxc758-feature-importance-visualization}}

PCA는 안쓰이는 컬럼을 버린다기 보다, 데이터 손실 최소화하여 차원을
줄이고 feature의 중요성을 확인할 수 있다.

    \begin{tcolorbox}[breakable, size=fbox, boxrule=1pt, pad at break*=1mm,colback=cellbackground, colframe=cellborder]
\prompt{In}{incolor}{1}{\boxspacing}
\begin{Verbatim}[commandchars=\\\{\}]
\PY{c+c1}{\PYZsh{}데이터 확인}

\PY{k+kn}{from} \PY{n+nn}{sklearn}\PY{n+nn}{.}\PY{n+nn}{datasets} \PY{k+kn}{import} \PY{n}{load\PYZus{}breast\PYZus{}cancer}
\PY{k+kn}{import} \PY{n+nn}{numpy} \PY{k}{as} \PY{n+nn}{np}
\PY{k+kn}{import} \PY{n+nn}{pandas} \PY{k}{as} \PY{n+nn}{pd}


\PY{n}{raw\PYZus{}data} \PY{o}{=} \PY{n}{load\PYZus{}breast\PYZus{}cancer}\PY{p}{(}\PY{p}{)}
\PY{n+nb}{print}\PY{p}{(}\PY{l+s+s1}{\PYZsq{}}\PY{l+s+s1}{key 값들 }\PY{l+s+s1}{\PYZsq{}}\PY{p}{,}\PY{n+nb}{dir}\PY{p}{(}\PY{n}{raw\PYZus{}data}\PY{p}{)}\PY{p}{)}\PY{c+c1}{\PYZsh{} raw data가 딕셔너리 형태라서 dir로 key값들 확인}
\PY{n+nb}{print}\PY{p}{(}\PY{l+s+s1}{\PYZsq{}}\PY{l+s+s1}{데이터 }\PY{l+s+s1}{\PYZsq{}}\PY{p}{,}\PY{n}{raw\PYZus{}data}\PY{o}{.}\PY{n}{data}\PY{o}{.}\PY{n}{shape}\PY{p}{)}
\PY{n+nb}{print}\PY{p}{(}\PY{l+s+s1}{\PYZsq{}}\PY{l+s+s1}{feature }\PY{l+s+s1}{\PYZsq{}}\PY{p}{,}\PY{n}{raw\PYZus{}data}\PY{o}{.}\PY{n}{feature\PYZus{}names}\PY{p}{)}
\PY{n}{DATA}\PY{o}{=}\PY{n}{pd}\PY{o}{.}\PY{n}{DataFrame}\PY{p}{(}\PY{n}{raw\PYZus{}data}\PY{o}{.}\PY{n}{data}\PY{p}{)}
\PY{n}{DATA}\PY{o}{.}\PY{n}{columns}\PY{o}{=}\PY{n}{raw\PYZus{}data}\PY{o}{.}\PY{n}{feature\PYZus{}names}
\PY{n}{DATA}\PY{o}{.}\PY{n}{head}\PY{p}{(}\PY{p}{)} \PY{c+c1}{\PYZsh{} pca는 데이터가 정규화가 필요할까?}
\PY{c+c1}{\PYZsh{}공분산은 데이터의 크기에 영향을 받기 때문에 해야된다고 생각.}
\end{Verbatim}
\end{tcolorbox}

    \begin{Verbatim}[commandchars=\\\{\}]
key 값들  ['DESCR', 'data', 'feature\_names', 'filename', 'frame', 'target',
'target\_names']
데이터  (569, 30)
feature  ['mean radius' 'mean texture' 'mean perimeter' 'mean area'
 'mean smoothness' 'mean compactness' 'mean concavity'
 'mean concave points' 'mean symmetry' 'mean fractal dimension'
 'radius error' 'texture error' 'perimeter error' 'area error'
 'smoothness error' 'compactness error' 'concavity error'
 'concave points error' 'symmetry error' 'fractal dimension error'
 'worst radius' 'worst texture' 'worst perimeter' 'worst area'
 'worst smoothness' 'worst compactness' 'worst concavity'
 'worst concave points' 'worst symmetry' 'worst fractal dimension']
    \end{Verbatim}

            \begin{tcolorbox}[breakable, size=fbox, boxrule=.5pt, pad at break*=1mm, opacityfill=0]
\prompt{Out}{outcolor}{1}{\boxspacing}
\begin{Verbatim}[commandchars=\\\{\}]
   mean radius  mean texture  mean perimeter  mean area  mean smoothness  \textbackslash{}
0        17.99         10.38          122.80     1001.0          0.11840
1        20.57         17.77          132.90     1326.0          0.08474
2        19.69         21.25          130.00     1203.0          0.10960
3        11.42         20.38           77.58      386.1          0.14250
4        20.29         14.34          135.10     1297.0          0.10030

   mean compactness  mean concavity  mean concave points  mean symmetry  \textbackslash{}
0           0.27760          0.3001              0.14710         0.2419
1           0.07864          0.0869              0.07017         0.1812
2           0.15990          0.1974              0.12790         0.2069
3           0.28390          0.2414              0.10520         0.2597
4           0.13280          0.1980              0.10430         0.1809

   mean fractal dimension  {\ldots}  worst radius  worst texture  worst perimeter  \textbackslash{}
0                 0.07871  {\ldots}         25.38          17.33           184.60
1                 0.05667  {\ldots}         24.99          23.41           158.80
2                 0.05999  {\ldots}         23.57          25.53           152.50
3                 0.09744  {\ldots}         14.91          26.50            98.87
4                 0.05883  {\ldots}         22.54          16.67           152.20

   worst area  worst smoothness  worst compactness  worst concavity  \textbackslash{}
0      2019.0            0.1622             0.6656           0.7119
1      1956.0            0.1238             0.1866           0.2416
2      1709.0            0.1444             0.4245           0.4504
3       567.7            0.2098             0.8663           0.6869
4      1575.0            0.1374             0.2050           0.4000

   worst concave points  worst symmetry  worst fractal dimension
0                0.2654          0.4601                  0.11890
1                0.1860          0.2750                  0.08902
2                0.2430          0.3613                  0.08758
3                0.2575          0.6638                  0.17300
4                0.1625          0.2364                  0.07678

[5 rows x 30 columns]
\end{Verbatim}
\end{tcolorbox}
        
    \hypertarget{uxb370uxc774uxd130-uxc815uxaddcuxd654}{%
\subsubsection{데이터
정규화}\label{uxb370uxc774uxd130-uxc815uxaddcuxd654}}

\hypertarget{min-max-scaler-uxc774uxc6a9}{%
\paragraph{min-max scaler 이용}\label{min-max-scaler-uxc774uxc6a9}}

최대값으로 나눠도 되지만,

값을 0\textasciitilde1 비율로 바꿔주기 위해, 분자에서는 최소값을 빼고

분모에서는 최대 - 최소를 해준다.

    \begin{tcolorbox}[breakable, size=fbox, boxrule=1pt, pad at break*=1mm,colback=cellbackground, colframe=cellborder]
\prompt{In}{incolor}{2}{\boxspacing}
\begin{Verbatim}[commandchars=\\\{\}]
\PY{k+kn}{from} \PY{n+nn}{sklearn}\PY{n+nn}{.}\PY{n+nn}{preprocessing} \PY{k+kn}{import} \PY{n}{MinMaxScaler}
\PY{n}{scaler} \PY{o}{=} \PY{n}{MinMaxScaler}\PY{p}{(}\PY{p}{)}
\PY{n}{DATA\PYZus{}norm} \PY{o}{=} \PY{n}{scaler}\PY{o}{.}\PY{n}{fit\PYZus{}transform}\PY{p}{(}\PY{n}{DATA}\PY{p}{[}\PY{p}{:}\PY{p}{]}\PY{p}{)}
\PY{n}{DATA\PYZus{}norm} \PY{o}{=} \PY{n}{pd}\PY{o}{.}\PY{n}{DataFrame}\PY{p}{(}\PY{n}{DATA\PYZus{}norm}\PY{p}{)}
\PY{n}{DATA\PYZus{}norm}\PY{o}{.}\PY{n}{columns} \PY{o}{=} \PY{n}{raw\PYZus{}data}\PY{o}{.}\PY{n}{feature\PYZus{}names}
\PY{n}{DATA\PYZus{}norm}\PY{o}{.}\PY{n}{head}\PY{p}{(}\PY{p}{)}
\end{Verbatim}
\end{tcolorbox}

            \begin{tcolorbox}[breakable, size=fbox, boxrule=.5pt, pad at break*=1mm, opacityfill=0]
\prompt{Out}{outcolor}{2}{\boxspacing}
\begin{Verbatim}[commandchars=\\\{\}]
   mean radius  mean texture  mean perimeter  mean area  mean smoothness  \textbackslash{}
0     0.521037      0.022658        0.545989   0.363733         0.593753
1     0.643144      0.272574        0.615783   0.501591         0.289880
2     0.601496      0.390260        0.595743   0.449417         0.514309
3     0.210090      0.360839        0.233501   0.102906         0.811321
4     0.629893      0.156578        0.630986   0.489290         0.430351

   mean compactness  mean concavity  mean concave points  mean symmetry  \textbackslash{}
0          0.792037        0.703140             0.731113       0.686364
1          0.181768        0.203608             0.348757       0.379798
2          0.431017        0.462512             0.635686       0.509596
3          0.811361        0.565604             0.522863       0.776263
4          0.347893        0.463918             0.518390       0.378283

   mean fractal dimension  {\ldots}  worst radius  worst texture  worst perimeter  \textbackslash{}
0                0.605518  {\ldots}      0.620776       0.141525         0.668310
1                0.141323  {\ldots}      0.606901       0.303571         0.539818
2                0.211247  {\ldots}      0.556386       0.360075         0.508442
3                1.000000  {\ldots}      0.248310       0.385928         0.241347
4                0.186816  {\ldots}      0.519744       0.123934         0.506948

   worst area  worst smoothness  worst compactness  worst concavity  \textbackslash{}
0    0.450698          0.601136           0.619292         0.568610
1    0.435214          0.347553           0.154563         0.192971
2    0.374508          0.483590           0.385375         0.359744
3    0.094008          0.915472           0.814012         0.548642
4    0.341575          0.437364           0.172415         0.319489

   worst concave points  worst symmetry  worst fractal dimension
0              0.912027        0.598462                 0.418864
1              0.639175        0.233590                 0.222878
2              0.835052        0.403706                 0.213433
3              0.884880        1.000000                 0.773711
4              0.558419        0.157500                 0.142595

[5 rows x 30 columns]
\end{Verbatim}
\end{tcolorbox}
        
    \hypertarget{d-space-uxcd95uxc18c}{%
\subsection{1. 2D SPACE 축소}\label{d-space-uxcd95uxc18c}}

변환 행렬 U와 데이터의 행렬곱을 통해, 데이터 축소.

    \begin{tcolorbox}[breakable, size=fbox, boxrule=1pt, pad at break*=1mm,colback=cellbackground, colframe=cellborder]
\prompt{In}{incolor}{267}{\boxspacing}
\begin{Verbatim}[commandchars=\\\{\}]
\PY{c+c1}{\PYZsh{} PCA 구현}
\PY{c+c1}{\PYZsh{} 1. 2D SPACE 축소}
\PY{k+kn}{from} \PY{n+nn}{matplotlib} \PY{k+kn}{import} \PY{n}{font\PYZus{}manager}\PY{p}{,} \PY{n}{rc}
\PY{n}{font\PYZus{}path} \PY{o}{=} \PY{l+s+s2}{\PYZdq{}}\PY{l+s+s2}{C:/Windows/Fonts/NGULIM.TTF}\PY{l+s+s2}{\PYZdq{}}
\PY{n}{font} \PY{o}{=} \PY{n}{font\PYZus{}manager}\PY{o}{.}\PY{n}{FontProperties}\PY{p}{(}\PY{n}{fname}\PY{o}{=}\PY{n}{font\PYZus{}path}\PY{p}{)}\PY{o}{.}\PY{n}{get\PYZus{}name}\PY{p}{(}\PY{p}{)}
\PY{n}{rc}\PY{p}{(}\PY{l+s+s1}{\PYZsq{}}\PY{l+s+s1}{font}\PY{l+s+s1}{\PYZsq{}}\PY{p}{,} \PY{n}{family}\PY{o}{=}\PY{n}{font}\PY{p}{)}

\PY{k+kn}{import} \PY{n+nn}{numpy} \PY{k}{as} \PY{n+nn}{np}
\PY{k+kn}{import} \PY{n+nn}{pandas} \PY{k}{as} \PY{n+nn}{pd}
\PY{k+kn}{from} \PY{n+nn}{matplotlib} \PY{k+kn}{import} \PY{n}{pyplot} \PY{k}{as} \PY{n}{plt}


\PY{n}{DATA\PYZus{}COV}\PY{o}{=}\PY{n}{np}\PY{o}{.}\PY{n}{cov}\PY{p}{(}\PY{n}{DATA\PYZus{}norm}\PY{o}{.}\PY{n}{T}\PY{p}{)}
\PY{n+nb}{print}\PY{p}{(}\PY{l+s+s1}{\PYZsq{}}\PY{l+s+s1}{eigenvector 수: }\PY{l+s+s1}{\PYZsq{}}\PY{p}{,}\PY{n}{DATA\PYZus{}COV}\PY{o}{.}\PY{n}{shape}\PY{p}{[}\PY{l+m+mi}{1}\PY{p}{]}\PY{p}{)}

\PY{c+c1}{\PYZsh{}decomposition 후, eigen value가 높은 두 벡터 선택.}
\PY{c+c1}{\PYZsh{}\PYZsh{} 추후, 자동으로 두개 선택하도록 고쳐야함.}
\PY{n}{eig\PYZus{}vals}\PY{p}{,} \PY{n}{eig\PYZus{}vecs}\PY{o}{=}\PY{n}{np}\PY{o}{.}\PY{n}{linalg}\PY{o}{.}\PY{n}{eig}\PY{p}{(}\PY{n}{DATA\PYZus{}COV}\PY{p}{)}
\PY{n+nb}{print}\PY{p}{(}\PY{l+s+s2}{\PYZdq{}}\PY{l+s+s2}{데이터 유지 비율: }\PY{l+s+s2}{\PYZdq{}}\PY{p}{,} \PY{n}{np}\PY{o}{.}\PY{n}{sum}\PY{p}{(}\PY{n}{eig\PYZus{}vals}\PY{p}{[}\PY{l+m+mi}{0}\PY{p}{:}\PY{l+m+mi}{2}\PY{p}{]}\PY{p}{)}\PY{o}{/}\PY{n}{np}\PY{o}{.}\PY{n}{sum}\PY{p}{(}\PY{n}{eig\PYZus{}vals}\PY{p}{)}\PY{p}{)}
\PY{n}{U}\PY{o}{=}\PY{n}{eig\PYZus{}vecs}\PY{p}{[}\PY{p}{:}\PY{p}{,}\PY{l+m+mi}{0}\PY{p}{:}\PY{l+m+mi}{2}\PY{p}{]}

\PY{c+c1}{\PYZsh{}데이터 사영}
\PY{n}{DATA\PYZus{}PCA}\PY{o}{=}\PY{n}{DATA\PYZus{}norm}\PY{o}{.}\PY{n}{dot}\PY{p}{(}\PY{n}{U}\PY{p}{)}


\PY{n}{DATA\PYZus{}PCA}\PY{o}{.}\PY{n}{columns}\PY{o}{=}\PY{p}{[}\PY{l+s+s1}{\PYZsq{}}\PY{l+s+s1}{PCA1}\PY{l+s+s1}{\PYZsq{}}\PY{p}{,}\PY{l+s+s1}{\PYZsq{}}\PY{l+s+s1}{PCA2}\PY{l+s+s1}{\PYZsq{}}\PY{p}{]}

\PY{n}{fig} \PY{o}{=}\PY{n}{plt}\PY{o}{.}\PY{n}{figure}\PY{p}{(}\PY{n}{figsize}\PY{o}{=}\PY{p}{(}\PY{l+m+mi}{8}\PY{p}{,}\PY{l+m+mi}{6}\PY{p}{)}\PY{p}{)}
\PY{n}{benign}\PY{o}{=} \PY{n}{DATA\PYZus{}PCA}\PY{p}{[}\PY{n}{raw\PYZus{}data}\PY{p}{[}\PY{l+s+s1}{\PYZsq{}}\PY{l+s+s1}{target}\PY{l+s+s1}{\PYZsq{}}\PY{p}{]}\PY{o}{==}\PY{l+m+mi}{1}\PY{p}{]} \PY{c+c1}{\PYZsh{}양성}
\PY{n}{malignant} \PY{o}{=} \PY{n}{DATA\PYZus{}PCA}\PY{p}{[}\PY{n}{raw\PYZus{}data}\PY{p}{[}\PY{l+s+s1}{\PYZsq{}}\PY{l+s+s1}{target}\PY{l+s+s1}{\PYZsq{}}\PY{p}{]}\PY{o}{==}\PY{l+m+mi}{0}\PY{p}{]} \PY{c+c1}{\PYZsh{}악성}

\PY{n}{plt}\PY{o}{.}\PY{n}{scatter}\PY{p}{(}\PY{n}{benign}\PY{o}{.}\PY{n}{iloc}\PY{p}{[}\PY{p}{:}\PY{p}{,}\PY{l+m+mi}{0}\PY{p}{]}\PY{p}{,}\PY{n}{benign}\PY{o}{.}\PY{n}{iloc}\PY{p}{[}\PY{p}{:}\PY{p}{,}\PY{l+m+mi}{1}\PY{p}{]}\PY{p}{,}\PY{n}{label}\PY{o}{=}\PY{l+s+s1}{\PYZsq{}}\PY{l+s+s1}{양성}\PY{l+s+s1}{\PYZsq{}}\PY{p}{)}
\PY{n}{plt}\PY{o}{.}\PY{n}{scatter}\PY{p}{(}\PY{n}{malignant}\PY{o}{.}\PY{n}{iloc}\PY{p}{[}\PY{p}{:}\PY{p}{,}\PY{l+m+mi}{0}\PY{p}{]}\PY{p}{,}\PY{n}{malignant}\PY{o}{.}\PY{n}{iloc}\PY{p}{[}\PY{p}{:}\PY{p}{,}\PY{l+m+mi}{1}\PY{p}{]}\PY{p}{,}\PY{n}{label}\PY{o}{=}\PY{l+s+s1}{\PYZsq{}}\PY{l+s+s1}{악성}\PY{l+s+s1}{\PYZsq{}}\PY{p}{)}

\PY{n}{plt}\PY{o}{.}\PY{n}{xlabel}\PY{p}{(}\PY{l+s+s1}{\PYZsq{}}\PY{l+s+s1}{PCA FIRST COMPONENT}\PY{l+s+s1}{\PYZsq{}}\PY{p}{)}
\PY{n}{plt}\PY{o}{.}\PY{n}{ylabel}\PY{p}{(}\PY{l+s+s1}{\PYZsq{}}\PY{l+s+s1}{PCA SECOND COMPONENT}\PY{l+s+s1}{\PYZsq{}}\PY{p}{)}
\PY{n}{plt}\PY{o}{.}\PY{n}{legend}\PY{p}{(}\PY{p}{)}
\PY{n}{plt}\PY{o}{.}\PY{n}{show}\PY{p}{(}\PY{p}{)}

\PY{c+c1}{\PYZsh{} eigen vector는 상수배를 하면 달라지기 때문에, 방향이 다를 수 도 있다.}
\end{Verbatim}
\end{tcolorbox}

    \begin{Verbatim}[commandchars=\\\{\}]
eigenvector 수:  30
데이터 유지 비율:  0.7038117901347682
    \end{Verbatim}

    \begin{center}
    \adjustimage{max size={0.9\linewidth}{0.9\paperheight}}{breast_cancer_pca_files/breast_cancer_pca_5_1.png}
    \end{center}
    { \hspace*{\fill} \\}
    
    \begin{tcolorbox}[breakable, size=fbox, boxrule=1pt, pad at break*=1mm,colback=cellbackground, colframe=cellborder]
\prompt{In}{incolor}{261}{\boxspacing}
\begin{Verbatim}[commandchars=\\\{\}]
\PY{n}{DATA\PYZus{}PCA}\PY{o}{.}\PY{n}{head}\PY{p}{(}\PY{p}{)}
\end{Verbatim}
\end{tcolorbox}

            \begin{tcolorbox}[breakable, size=fbox, boxrule=.5pt, pad at break*=1mm, opacityfill=0]
\prompt{Out}{outcolor}{261}{\boxspacing}
\begin{Verbatim}[commandchars=\\\{\}]
       PCA1      PCA2
0 -2.590770  0.762974
1 -1.666057 -0.220868
2 -2.158371  0.226378
3 -2.204565  1.861168
4 -1.830577  0.033608
\end{Verbatim}
\end{tcolorbox}
        
    \hypertarget{pca-componentsuxc758-feature-importance-visualization}{%
\subsection{2. PCA components의 feature importance
visualization}\label{pca-componentsuxc758-feature-importance-visualization}}

U행렬. 주성분의 값을 활용해, feature importance 확인.

    \begin{tcolorbox}[breakable, size=fbox, boxrule=1pt, pad at break*=1mm,colback=cellbackground, colframe=cellborder]
\prompt{In}{incolor}{268}{\boxspacing}
\begin{Verbatim}[commandchars=\\\{\}]
\PY{c+c1}{\PYZsh{}feature visualization}
\PY{k+kn}{import} \PY{n+nn}{seaborn} \PY{k}{as} \PY{n+nn}{sns}

\PY{n}{U\PYZus{}df}\PY{o}{=}\PY{n}{pd}\PY{o}{.}\PY{n}{DataFrame}\PY{p}{(}\PY{n}{U}\PY{o}{.}\PY{n}{T}\PY{p}{)}
\PY{n}{U\PYZus{}df}\PY{o}{.}\PY{n}{columns}\PY{o}{=}\PY{n}{raw\PYZus{}data}\PY{o}{.}\PY{n}{feature\PYZus{}names}
\PY{n}{U\PYZus{}df}\PY{o}{.}\PY{n}{index}\PY{o}{=}\PY{p}{[}\PY{l+s+s1}{\PYZsq{}}\PY{l+s+s1}{PCA1}\PY{l+s+s1}{\PYZsq{}}\PY{p}{,}\PY{l+s+s1}{\PYZsq{}}\PY{l+s+s1}{PCA2}\PY{l+s+s1}{\PYZsq{}}\PY{p}{]}
\PY{n}{U\PYZus{}df}

\PY{n}{fig} \PY{o}{=}\PY{n}{plt}\PY{o}{.}\PY{n}{figure}\PY{p}{(}\PY{n}{figsize}\PY{o}{=}\PY{p}{(}\PY{l+m+mi}{11}\PY{p}{,}\PY{l+m+mi}{11}\PY{p}{)}\PY{p}{)}

\PY{n}{plt}\PY{o}{.}\PY{n}{subplot}\PY{p}{(}\PY{l+m+mi}{221}\PY{p}{)}\PY{p}{,}\PY{n}{plt}\PY{o}{.}\PY{n}{title}\PY{p}{(}\PY{l+s+s2}{\PYZdq{}}\PY{l+s+s2}{pca1}\PY{l+s+s2}{\PYZdq{}}\PY{p}{)}\PY{c+c1}{\PYZsh{} 2행 2열중 1번째}
\PY{n}{plt}\PY{o}{.}\PY{n}{bar}\PY{p}{(}\PY{n}{np}\PY{o}{.}\PY{n}{arange}\PY{p}{(}\PY{l+m+mi}{30}\PY{p}{)}\PY{p}{,}\PY{n}{U\PYZus{}df}\PY{o}{.}\PY{n}{loc}\PY{p}{[}\PY{l+s+s1}{\PYZsq{}}\PY{l+s+s1}{PCA1}\PY{l+s+s1}{\PYZsq{}}\PY{p}{,}\PY{p}{:}\PY{p}{]}\PY{p}{)}

\PY{n}{plt}\PY{o}{.}\PY{n}{subplot}\PY{p}{(}\PY{l+m+mi}{222}\PY{p}{)}\PY{p}{,}\PY{n}{plt}\PY{o}{.}\PY{n}{title}\PY{p}{(}\PY{l+s+s2}{\PYZdq{}}\PY{l+s+s2}{pca2}\PY{l+s+s2}{\PYZdq{}}\PY{p}{)}\PY{c+c1}{\PYZsh{} 2행 2열중 1번째}
\PY{n}{plt}\PY{o}{.}\PY{n}{bar}\PY{p}{(}\PY{n}{np}\PY{o}{.}\PY{n}{arange}\PY{p}{(}\PY{l+m+mi}{30}\PY{p}{)}\PY{p}{,}\PY{n}{U\PYZus{}df}\PY{o}{.}\PY{n}{loc}\PY{p}{[}\PY{l+s+s1}{\PYZsq{}}\PY{l+s+s1}{PCA2}\PY{l+s+s1}{\PYZsq{}}\PY{p}{,}\PY{p}{:}\PY{p}{]}\PY{p}{)}

\PY{n}{plt}\PY{o}{.}\PY{n}{show}\PY{p}{(}\PY{p}{)}
\end{Verbatim}
\end{tcolorbox}

    \begin{center}
    \adjustimage{max size={0.9\linewidth}{0.9\paperheight}}{breast_cancer_pca_files/breast_cancer_pca_8_0.png}
    \end{center}
    { \hspace*{\fill} \\}
    
    \begin{tcolorbox}[breakable, size=fbox, boxrule=1pt, pad at break*=1mm,colback=cellbackground, colframe=cellborder]
\prompt{In}{incolor}{264}{\boxspacing}
\begin{Verbatim}[commandchars=\\\{\}]
\PY{n}{U\PYZus{}df}
\end{Verbatim}
\end{tcolorbox}

            \begin{tcolorbox}[breakable, size=fbox, boxrule=.5pt, pad at break*=1mm, opacityfill=0]
\prompt{Out}{outcolor}{264}{\boxspacing}
\begin{Verbatim}[commandchars=\\\{\}]
      mean radius  mean texture  mean perimeter  mean area  mean smoothness  \textbackslash{}
PCA1    -0.242676     -0.096479        -0.25255  -0.216495        -0.109695
PCA2    -0.261317     -0.059058        -0.23859  -0.231107         0.199884

      mean compactness  mean concavity  mean concave points  mean symmetry  \textbackslash{}
PCA1         -0.240398       -0.301914            -0.322475      -0.111432
PCA2          0.213915        0.113811            -0.008312       0.211115

      mean fractal dimension  {\ldots}  worst radius  worst texture  \textbackslash{}
PCA1               -0.043298  {\ldots}     -0.259387      -0.113833
PCA2                0.406392  {\ldots}     -0.244282      -0.039682

      worst perimeter  worst area  worst smoothness  worst compactness  \textbackslash{}
PCA1        -0.260708   -0.205918         -0.123774          -0.204993
PCA2        -0.211635   -0.198979          0.229455           0.198064

      worst concavity  worst concave points  worst symmetry  \textbackslash{}
PCA1        -0.244084             -0.371065       -0.095923
PCA2         0.154982              0.044329        0.151169

      worst fractal dimension
PCA1                -0.094634
PCA2                 0.258092

[2 rows x 30 columns]
\end{Verbatim}
\end{tcolorbox}
        
    \begin{tcolorbox}[breakable, size=fbox, boxrule=1pt, pad at break*=1mm,colback=cellbackground, colframe=cellborder]
\prompt{In}{incolor}{269}{\boxspacing}
\begin{Verbatim}[commandchars=\\\{\}]
\PY{n}{fig} \PY{o}{=}\PY{n}{plt}\PY{o}{.}\PY{n}{figure}\PY{p}{(}\PY{n}{figsize}\PY{o}{=}\PY{p}{(}\PY{l+m+mi}{10}\PY{p}{,}\PY{l+m+mi}{6}\PY{p}{)}\PY{p}{)}
\PY{n}{ax} \PY{o}{=} \PY{n}{sns}\PY{o}{.}\PY{n}{heatmap}\PY{p}{(}\PY{n}{U\PYZus{}df}\PY{p}{,}
                 \PY{n}{cmap}\PY{o}{=}\PY{l+s+s1}{\PYZsq{}}\PY{l+s+s1}{RdYlGn\PYZus{}r}\PY{l+s+s1}{\PYZsq{}}\PY{p}{,}\PY{c+c1}{\PYZsh{} cmap Color}
                 \PY{n}{annot}\PY{o}{=}\PY{k+kc}{False}\PY{p}{,}            \PY{c+c1}{\PYZsh{} Value Text}
                 \PY{n}{square}\PY{o}{=}\PY{k+kc}{True}\PY{p}{,}
                \PY{p}{)}
\PY{c+c1}{\PYZsh{}PCA1과 PCA2 모두 각자 다른 feature가 높다.}
\end{Verbatim}
\end{tcolorbox}

    \begin{center}
    \adjustimage{max size={0.9\linewidth}{0.9\paperheight}}{breast_cancer_pca_files/breast_cancer_pca_10_0.png}
    \end{center}
    { \hspace*{\fill} \\}
    
    \begin{center}\rule{0.5\linewidth}{0.5pt}\end{center}

PCA 라이브러리 활용

    \begin{tcolorbox}[breakable, size=fbox, boxrule=1pt, pad at break*=1mm,colback=cellbackground, colframe=cellborder]
\prompt{In}{incolor}{229}{\boxspacing}
\begin{Verbatim}[commandchars=\\\{\}]
\PY{c+c1}{\PYZsh{} pca 라이브러리 활용.}
\PY{k+kn}{from} \PY{n+nn}{sklearn}\PY{n+nn}{.}\PY{n+nn}{preprocessing} \PY{k+kn}{import} \PY{n}{MinMaxScaler}
\PY{k+kn}{from} \PY{n+nn}{sklearn}\PY{n+nn}{.}\PY{n+nn}{decomposition} \PY{k+kn}{import} \PY{n}{PCA}

\PY{n}{df2} \PY{o}{=} \PY{n}{pd}\PY{o}{.}\PY{n}{DataFrame}\PY{p}{(}\PY{n}{raw\PYZus{}data}\PY{p}{[}\PY{l+s+s1}{\PYZsq{}}\PY{l+s+s1}{data}\PY{l+s+s1}{\PYZsq{}}\PY{p}{]}\PY{p}{,}\PY{n}{columns}\PY{o}{=}\PY{n}{raw\PYZus{}data}\PY{p}{[}\PY{l+s+s1}{\PYZsq{}}\PY{l+s+s1}{feature\PYZus{}names}\PY{l+s+s1}{\PYZsq{}}\PY{p}{]}\PY{p}{)}
\PY{n}{scaler}\PY{o}{=}\PY{n}{MinMaxScaler}\PY{p}{(}\PY{p}{)}
\PY{n}{scaler}\PY{o}{.}\PY{n}{fit}\PY{p}{(}\PY{n}{df2}\PY{p}{)}
\PY{n}{scaled\PYZus{}data}\PY{o}{=}\PY{n}{scaler}\PY{o}{.}\PY{n}{transform}\PY{p}{(}\PY{n}{df2}\PY{p}{)}
\PY{n}{pca2}\PY{o}{=}\PY{n}{PCA}\PY{p}{(}\PY{n}{n\PYZus{}components}\PY{o}{=}\PY{l+m+mi}{2}\PY{p}{)}
\PY{n}{pca2}\PY{o}{.}\PY{n}{fit}\PY{p}{(}\PY{n}{scaled\PYZus{}data}\PY{p}{)}

\PY{n}{pca2\PYZus{}data}\PY{o}{=}\PY{n}{pca2}\PY{o}{.}\PY{n}{transform}\PY{p}{(}\PY{n}{scaled\PYZus{}data}\PY{p}{)}
\PY{n}{pca2\PYZus{}df}\PY{o}{=}\PY{n}{pd}\PY{o}{.}\PY{n}{DataFrame}\PY{p}{(}\PY{n}{pca2\PYZus{}data}\PY{p}{,}\PY{n}{columns}\PY{o}{=}\PY{p}{[}\PY{l+s+s1}{\PYZsq{}}\PY{l+s+s1}{PCA1}\PY{l+s+s1}{\PYZsq{}}\PY{p}{,}\PY{l+s+s1}{\PYZsq{}}\PY{l+s+s1}{PCA2}\PY{l+s+s1}{\PYZsq{}}\PY{p}{]}\PY{p}{)}

\PY{n}{fig} \PY{o}{=}\PY{n}{plt}\PY{o}{.}\PY{n}{figure}\PY{p}{(}\PY{n}{figsize}\PY{o}{=}\PY{p}{(}\PY{l+m+mi}{8}\PY{p}{,}\PY{l+m+mi}{6}\PY{p}{)}\PY{p}{)}
\PY{n}{benign2}\PY{o}{=} \PY{n}{pca2\PYZus{}df}\PY{p}{[}\PY{n}{raw\PYZus{}data}\PY{p}{[}\PY{l+s+s1}{\PYZsq{}}\PY{l+s+s1}{target}\PY{l+s+s1}{\PYZsq{}}\PY{p}{]}\PY{o}{==}\PY{l+m+mi}{1}\PY{p}{]} \PY{c+c1}{\PYZsh{}양성}
\PY{n}{malignant2} \PY{o}{=} \PY{n}{pca2\PYZus{}df}\PY{p}{[}\PY{n}{raw\PYZus{}data}\PY{p}{[}\PY{l+s+s1}{\PYZsq{}}\PY{l+s+s1}{target}\PY{l+s+s1}{\PYZsq{}}\PY{p}{]}\PY{o}{==}\PY{l+m+mi}{0}\PY{p}{]} \PY{c+c1}{\PYZsh{}악성}

\PY{n}{plt}\PY{o}{.}\PY{n}{scatter}\PY{p}{(}\PY{n}{benign2}\PY{o}{.}\PY{n}{iloc}\PY{p}{[}\PY{p}{:}\PY{p}{,}\PY{l+m+mi}{0}\PY{p}{]}\PY{p}{,}\PY{n}{benign2}\PY{o}{.}\PY{n}{iloc}\PY{p}{[}\PY{p}{:}\PY{p}{,}\PY{l+m+mi}{1}\PY{p}{]}\PY{p}{,}\PY{n}{label}\PY{o}{=}\PY{l+s+s1}{\PYZsq{}}\PY{l+s+s1}{양성}\PY{l+s+s1}{\PYZsq{}}\PY{p}{)}
\PY{n}{plt}\PY{o}{.}\PY{n}{scatter}\PY{p}{(}\PY{n}{malignant2}\PY{o}{.}\PY{n}{iloc}\PY{p}{[}\PY{p}{:}\PY{p}{,}\PY{l+m+mi}{0}\PY{p}{]}\PY{p}{,}\PY{n}{malignant2}\PY{o}{.}\PY{n}{iloc}\PY{p}{[}\PY{p}{:}\PY{p}{,}\PY{l+m+mi}{1}\PY{p}{]}\PY{p}{,}\PY{n}{label}\PY{o}{=}\PY{l+s+s1}{\PYZsq{}}\PY{l+s+s1}{악성}\PY{l+s+s1}{\PYZsq{}}\PY{p}{)}

\PY{n}{plt}\PY{o}{.}\PY{n}{xlabel}\PY{p}{(}\PY{l+s+s1}{\PYZsq{}}\PY{l+s+s1}{PCA FIRST COMPONENT}\PY{l+s+s1}{\PYZsq{}}\PY{p}{)}
\PY{n}{plt}\PY{o}{.}\PY{n}{ylabel}\PY{p}{(}\PY{l+s+s1}{\PYZsq{}}\PY{l+s+s1}{PCA SECOND COMPONENT}\PY{l+s+s1}{\PYZsq{}}\PY{p}{)}
\PY{n}{plt}\PY{o}{.}\PY{n}{legend}\PY{p}{(}\PY{p}{)}
\PY{n}{plt}\PY{o}{.}\PY{n}{show}\PY{p}{(}\PY{p}{)}
\PY{n}{pca2\PYZus{}components\PYZus{}df}\PY{o}{=}\PY{n}{pd}\PY{o}{.}\PY{n}{DataFrame}\PY{p}{(}\PY{n}{pca2}\PY{o}{.}\PY{n}{components\PYZus{}}\PY{p}{,}\PY{n}{columns}\PY{o}{=}\PY{n}{raw\PYZus{}data}\PY{p}{[}\PY{l+s+s1}{\PYZsq{}}\PY{l+s+s1}{feature\PYZus{}names}\PY{l+s+s1}{\PYZsq{}}\PY{p}{]}\PY{p}{,}\PY{n}{index}\PY{o}{=}\PY{p}{[}\PY{l+s+s1}{\PYZsq{}}\PY{l+s+s1}{pca1}\PY{l+s+s1}{\PYZsq{}}\PY{p}{,}\PY{l+s+s1}{\PYZsq{}}\PY{l+s+s1}{pca2}\PY{l+s+s1}{\PYZsq{}}\PY{p}{]}\PY{p}{)}

\PY{n}{fig} \PY{o}{=}\PY{n}{plt}\PY{o}{.}\PY{n}{figure}\PY{p}{(}\PY{n}{figsize}\PY{o}{=}\PY{p}{(}\PY{l+m+mi}{10}\PY{p}{,}\PY{l+m+mi}{6}\PY{p}{)}\PY{p}{)}
\PY{n}{ax} \PY{o}{=} \PY{n}{sns}\PY{o}{.}\PY{n}{heatmap}\PY{p}{(}\PY{n}{pca2\PYZus{}components\PYZus{}df}\PY{p}{,}
                 \PY{n}{cmap}\PY{o}{=}\PY{l+s+s1}{\PYZsq{}}\PY{l+s+s1}{RdYlGn\PYZus{}r}\PY{l+s+s1}{\PYZsq{}}\PY{p}{,}\PY{c+c1}{\PYZsh{} cmap Color}
                 \PY{n}{annot}\PY{o}{=}\PY{k+kc}{False}\PY{p}{,}            \PY{c+c1}{\PYZsh{} Value Text}
                 \PY{n}{square}\PY{o}{=}\PY{k+kc}{True}\PY{p}{,}
                \PY{p}{)}
\end{Verbatim}
\end{tcolorbox}

    \begin{center}
    \adjustimage{max size={0.9\linewidth}{0.9\paperheight}}{breast_cancer_pca_files/breast_cancer_pca_12_0.png}
    \end{center}
    { \hspace*{\fill} \\}
    
    \begin{center}
    \adjustimage{max size={0.9\linewidth}{0.9\paperheight}}{breast_cancer_pca_files/breast_cancer_pca_12_1.png}
    \end{center}
    { \hspace*{\fill} \\}
    
    \begin{tcolorbox}[breakable, size=fbox, boxrule=1pt, pad at break*=1mm,colback=cellbackground, colframe=cellborder]
\prompt{In}{incolor}{230}{\boxspacing}
\begin{Verbatim}[commandchars=\\\{\}]
\PY{n}{pca2\PYZus{}df}\PY{o}{.}\PY{n}{head}\PY{p}{(}\PY{l+m+mi}{5}\PY{p}{)}
\end{Verbatim}
\end{tcolorbox}

            \begin{tcolorbox}[breakable, size=fbox, boxrule=.5pt, pad at break*=1mm, opacityfill=0]
\prompt{Out}{outcolor}{230}{\boxspacing}
\begin{Verbatim}[commandchars=\\\{\}]
       PCA1      PCA2
0  1.387021  0.426895
1  0.462308 -0.556947
2  0.954621 -0.109701
3  1.000816  1.525089
4  0.626828 -0.302471
\end{Verbatim}
\end{tcolorbox}
        
    \begin{tcolorbox}[breakable, size=fbox, boxrule=1pt, pad at break*=1mm,colback=cellbackground, colframe=cellborder]
\prompt{In}{incolor}{11}{\boxspacing}
\begin{Verbatim}[commandchars=\\\{\}]
\PY{n+nb}{print}\PY{p}{(}\PY{l+s+s1}{\PYZsq{}}\PY{l+s+s1}{데이터 }\PY{l+s+s1}{\PYZsq{}}\PY{p}{,}\PY{n}{raw\PYZus{}data}\PY{o}{.}\PY{n}{DESCR}\PY{p}{)}
\end{Verbatim}
\end{tcolorbox}

    \begin{Verbatim}[commandchars=\\\{\}]
데이터  .. \_breast\_cancer\_dataset:

Breast cancer wisconsin (diagnostic) dataset
--------------------------------------------

**Data Set Characteristics:**

    :Number of Instances: 569

    :Number of Attributes: 30 numeric, predictive attributes and the class

    :Attribute Information:
        - radius (mean of distances from center to points on the perimeter)
        - texture (standard deviation of gray-scale values)
        - perimeter
        - area
        - smoothness (local variation in radius lengths)
        - compactness (perimeter\^{}2 / area - 1.0)
        - concavity (severity of concave portions of the contour)
        - concave points (number of concave portions of the contour)
        - symmetry
        - fractal dimension ("coastline approximation" - 1)

        The mean, standard error, and "worst" or largest (mean of the three
        worst/largest values) of these features were computed for each image,
        resulting in 30 features.  For instance, field 0 is Mean Radius, field
        10 is Radius SE, field 20 is Worst Radius.

        - class:
                - WDBC-Malignant
                - WDBC-Benign

    :Summary Statistics:

    ===================================== ====== ======
                                           Min    Max
    ===================================== ====== ======
    radius (mean):                        6.981  28.11
    texture (mean):                       9.71   39.28
    perimeter (mean):                     43.79  188.5
    area (mean):                          143.5  2501.0
    smoothness (mean):                    0.053  0.163
    compactness (mean):                   0.019  0.345
    concavity (mean):                     0.0    0.427
    concave points (mean):                0.0    0.201
    symmetry (mean):                      0.106  0.304
    fractal dimension (mean):             0.05   0.097
    radius (standard error):              0.112  2.873
    texture (standard error):             0.36   4.885
    perimeter (standard error):           0.757  21.98
    area (standard error):                6.802  542.2
    smoothness (standard error):          0.002  0.031
    compactness (standard error):         0.002  0.135
    concavity (standard error):           0.0    0.396
    concave points (standard error):      0.0    0.053
    symmetry (standard error):            0.008  0.079
    fractal dimension (standard error):   0.001  0.03
    radius (worst):                       7.93   36.04
    texture (worst):                      12.02  49.54
    perimeter (worst):                    50.41  251.2
    area (worst):                         185.2  4254.0
    smoothness (worst):                   0.071  0.223
    compactness (worst):                  0.027  1.058
    concavity (worst):                    0.0    1.252
    concave points (worst):               0.0    0.291
    symmetry (worst):                     0.156  0.664
    fractal dimension (worst):            0.055  0.208
    ===================================== ====== ======

    :Missing Attribute Values: None

    :Class Distribution: 212 - Malignant, 357 - Benign

    :Creator:  Dr. William H. Wolberg, W. Nick Street, Olvi L. Mangasarian

    :Donor: Nick Street

    :Date: November, 1995

This is a copy of UCI ML Breast Cancer Wisconsin (Diagnostic) datasets.
https://goo.gl/U2Uwz2

Features are computed from a digitized image of a fine needle
aspirate (FNA) of a breast mass.  They describe
characteristics of the cell nuclei present in the image.

Separating plane described above was obtained using
Multisurface Method-Tree (MSM-T) [K. P. Bennett, "Decision Tree
Construction Via Linear Programming." Proceedings of the 4th
Midwest Artificial Intelligence and Cognitive Science Society,
pp. 97-101, 1992], a classification method which uses linear
programming to construct a decision tree.  Relevant features
were selected using an exhaustive search in the space of 1-4
features and 1-3 separating planes.

The actual linear program used to obtain the separating plane
in the 3-dimensional space is that described in:
[K. P. Bennett and O. L. Mangasarian: "Robust Linear
Programming Discrimination of Two Linearly Inseparable Sets",
Optimization Methods and Software 1, 1992, 23-34].

This database is also available through the UW CS ftp server:

ftp ftp.cs.wisc.edu
cd math-prog/cpo-dataset/machine-learn/WDBC/

.. topic:: References

   - W.N. Street, W.H. Wolberg and O.L. Mangasarian. Nuclear feature extraction
     for breast tumor diagnosis. IS\&T/SPIE 1993 International Symposium on
     Electronic Imaging: Science and Technology, volume 1905, pages 861-870,
     San Jose, CA, 1993.
   - O.L. Mangasarian, W.N. Street and W.H. Wolberg. Breast cancer diagnosis and
     prognosis via linear programming. Operations Research, 43(4), pages
570-577,
     July-August 1995.
   - W.H. Wolberg, W.N. Street, and O.L. Mangasarian. Machine learning
techniques
     to diagnose breast cancer from fine-needle aspirates. Cancer Letters 77
(1994)
     163-171.
    \end{Verbatim}

    \begin{center}\rule{0.5\linewidth}{0.5pt}\end{center}

참조.

https://losskatsu.github.io/machine-learning/sklearn/\#\%EA\%B0\%80\%EC\%83\%81-\%EB\%8D\%B0\%EC\%9D\%B4\%ED\%84\%B0-\%EB\%9E\%9C\%EB\%8D\%A4\%EC\%9C\%BC\%EB\%A1\%9C-\%EC\%83\%9D\%EC\%84\%B1\%ED\%95\%98\%EA\%B8\%B0

https://scikit-learn.org/stable/modules/generated/sklearn.datasets.load\_breast\_cancer.html

https://jfun.tistory.com/108

https://blog.naver.com/pmw9440/221843085568 : pca의 역할

https://blog.naver.com/pmw9440/221843886378 :왜 정규화를 해야하는지. 각
원소별 load

https://soo-jjeong.tistory.com/122 :min-max scaler 설명

https://www.youtube.com/watch?v=QdBy02ExhGI : 시각화
https://rfriend.tistory.com/419 : sns heatmap의 cmap 종류들


    % Add a bibliography block to the postdoc
    
    
    
\end{document}
