\documentclass[11pt]{article}

    \usepackage[breakable]{tcolorbox}
    \usepackage{parskip} % Stop auto-indenting (to mimic markdown behaviour)
     	
    \usepackage{iftex}
    \ifPDFTeX
    	\usepackage[T1]{fontenc}
    	\usepackage{mathpazo}
    \else
    	\usepackage{fontspec}
    \fi

    % Basic figure setup, for now with no caption control since it's done
    % automatically by Pandoc (which extracts ![](path) syntax from Markdown).
    \usepackage{graphicx}
    % Maintain compatibility with old templates. Remove in nbconvert 6.0
    \let\Oldincludegraphics\includegraphics
    % Ensure that by default, figures have no caption (until we provide a
    % proper Figure object with a Caption API and a way to capture that
    % in the conversion process - todo).
    \usepackage{caption}
    \DeclareCaptionFormat{nocaption}{}
    \captionsetup{format=nocaption,aboveskip=0pt,belowskip=0pt}

    \usepackage{float}
    \floatplacement{figure}{H} % forces figures to be placed at the correct location
    \usepackage{xcolor} % Allow colors to be defined
    \usepackage{enumerate} % Needed for markdown enumerations to work
    \usepackage{geometry} % Used to adjust the document margins
    \usepackage{amsmath} % Equations
    \usepackage{amssymb} % Equations
    \usepackage{textcomp} % defines textquotesingle
    % Hack from http://tex.stackexchange.com/a/47451/13684:
    \AtBeginDocument{%
        \def\PYZsq{\textquotesingle}% Upright quotes in Pygmentized code
    }
    \usepackage{upquote} % Upright quotes for verbatim code
    \usepackage{eurosym} % defines \euro
    \usepackage[mathletters]{ucs} % Extended unicode (utf-8) support
    \usepackage{fancyvrb} % verbatim replacement that allows latex
    \usepackage{grffile} % extends the file name processing of package graphics 
                         % to support a larger range
    \makeatletter % fix for old versions of grffile with XeLaTeX
    \@ifpackagelater{grffile}{2019/11/01}
    {
      % Do nothing on new versions
    }
    {
      \def\Gread@@xetex#1{%
        \IfFileExists{"\Gin@base".bb}%
        {\Gread@eps{\Gin@base.bb}}%
        {\Gread@@xetex@aux#1}%
      }
    }
    \makeatother
    \usepackage[Export]{adjustbox} % Used to constrain images to a maximum size
    \adjustboxset{max size={0.9\linewidth}{0.9\paperheight}}

    % The hyperref package gives us a pdf with properly built
    % internal navigation ('pdf bookmarks' for the table of contents,
    % internal cross-reference links, web links for URLs, etc.)
    \usepackage{hyperref}
    % The default LaTeX title has an obnoxious amount of whitespace. By default,
    % titling removes some of it. It also provides customization options.
    \usepackage{titling}
    \usepackage{longtable} % longtable support required by pandoc >1.10
    \usepackage{booktabs}  % table support for pandoc > 1.12.2
    \usepackage[inline]{enumitem} % IRkernel/repr support (it uses the enumerate* environment)
    \usepackage[normalem]{ulem} % ulem is needed to support strikethroughs (\sout)
                                % normalem makes italics be italics, not underlines
    \usepackage{mathrsfs}
    

    
    % Colors for the hyperref package
    \definecolor{urlcolor}{rgb}{0,.145,.698}
    \definecolor{linkcolor}{rgb}{.71,0.21,0.01}
    \definecolor{citecolor}{rgb}{.12,.54,.11}

    % ANSI colors
    \definecolor{ansi-black}{HTML}{3E424D}
    \definecolor{ansi-black-intense}{HTML}{282C36}
    \definecolor{ansi-red}{HTML}{E75C58}
    \definecolor{ansi-red-intense}{HTML}{B22B31}
    \definecolor{ansi-green}{HTML}{00A250}
    \definecolor{ansi-green-intense}{HTML}{007427}
    \definecolor{ansi-yellow}{HTML}{DDB62B}
    \definecolor{ansi-yellow-intense}{HTML}{B27D12}
    \definecolor{ansi-blue}{HTML}{208FFB}
    \definecolor{ansi-blue-intense}{HTML}{0065CA}
    \definecolor{ansi-magenta}{HTML}{D160C4}
    \definecolor{ansi-magenta-intense}{HTML}{A03196}
    \definecolor{ansi-cyan}{HTML}{60C6C8}
    \definecolor{ansi-cyan-intense}{HTML}{258F8F}
    \definecolor{ansi-white}{HTML}{C5C1B4}
    \definecolor{ansi-white-intense}{HTML}{A1A6B2}
    \definecolor{ansi-default-inverse-fg}{HTML}{FFFFFF}
    \definecolor{ansi-default-inverse-bg}{HTML}{000000}

    % common color for the border for error outputs.
    \definecolor{outerrorbackground}{HTML}{FFDFDF}

    % commands and environments needed by pandoc snippets
    % extracted from the output of `pandoc -s`
    \providecommand{\tightlist}{%
      \setlength{\itemsep}{0pt}\setlength{\parskip}{0pt}}
    \DefineVerbatimEnvironment{Highlighting}{Verbatim}{commandchars=\\\{\}}
    % Add ',fontsize=\small' for more characters per line
    \newenvironment{Shaded}{}{}
    \newcommand{\KeywordTok}[1]{\textcolor[rgb]{0.00,0.44,0.13}{\textbf{{#1}}}}
    \newcommand{\DataTypeTok}[1]{\textcolor[rgb]{0.56,0.13,0.00}{{#1}}}
    \newcommand{\DecValTok}[1]{\textcolor[rgb]{0.25,0.63,0.44}{{#1}}}
    \newcommand{\BaseNTok}[1]{\textcolor[rgb]{0.25,0.63,0.44}{{#1}}}
    \newcommand{\FloatTok}[1]{\textcolor[rgb]{0.25,0.63,0.44}{{#1}}}
    \newcommand{\CharTok}[1]{\textcolor[rgb]{0.25,0.44,0.63}{{#1}}}
    \newcommand{\StringTok}[1]{\textcolor[rgb]{0.25,0.44,0.63}{{#1}}}
    \newcommand{\CommentTok}[1]{\textcolor[rgb]{0.38,0.63,0.69}{\textit{{#1}}}}
    \newcommand{\OtherTok}[1]{\textcolor[rgb]{0.00,0.44,0.13}{{#1}}}
    \newcommand{\AlertTok}[1]{\textcolor[rgb]{1.00,0.00,0.00}{\textbf{{#1}}}}
    \newcommand{\FunctionTok}[1]{\textcolor[rgb]{0.02,0.16,0.49}{{#1}}}
    \newcommand{\RegionMarkerTok}[1]{{#1}}
    \newcommand{\ErrorTok}[1]{\textcolor[rgb]{1.00,0.00,0.00}{\textbf{{#1}}}}
    \newcommand{\NormalTok}[1]{{#1}}
    
    % Additional commands for more recent versions of Pandoc
    \newcommand{\ConstantTok}[1]{\textcolor[rgb]{0.53,0.00,0.00}{{#1}}}
    \newcommand{\SpecialCharTok}[1]{\textcolor[rgb]{0.25,0.44,0.63}{{#1}}}
    \newcommand{\VerbatimStringTok}[1]{\textcolor[rgb]{0.25,0.44,0.63}{{#1}}}
    \newcommand{\SpecialStringTok}[1]{\textcolor[rgb]{0.73,0.40,0.53}{{#1}}}
    \newcommand{\ImportTok}[1]{{#1}}
    \newcommand{\DocumentationTok}[1]{\textcolor[rgb]{0.73,0.13,0.13}{\textit{{#1}}}}
    \newcommand{\AnnotationTok}[1]{\textcolor[rgb]{0.38,0.63,0.69}{\textbf{\textit{{#1}}}}}
    \newcommand{\CommentVarTok}[1]{\textcolor[rgb]{0.38,0.63,0.69}{\textbf{\textit{{#1}}}}}
    \newcommand{\VariableTok}[1]{\textcolor[rgb]{0.10,0.09,0.49}{{#1}}}
    \newcommand{\ControlFlowTok}[1]{\textcolor[rgb]{0.00,0.44,0.13}{\textbf{{#1}}}}
    \newcommand{\OperatorTok}[1]{\textcolor[rgb]{0.40,0.40,0.40}{{#1}}}
    \newcommand{\BuiltInTok}[1]{{#1}}
    \newcommand{\ExtensionTok}[1]{{#1}}
    \newcommand{\PreprocessorTok}[1]{\textcolor[rgb]{0.74,0.48,0.00}{{#1}}}
    \newcommand{\AttributeTok}[1]{\textcolor[rgb]{0.49,0.56,0.16}{{#1}}}
    \newcommand{\InformationTok}[1]{\textcolor[rgb]{0.38,0.63,0.69}{\textbf{\textit{{#1}}}}}
    \newcommand{\WarningTok}[1]{\textcolor[rgb]{0.38,0.63,0.69}{\textbf{\textit{{#1}}}}}
    
    
    % Define a nice break command that doesn't care if a line doesn't already
    % exist.
    \def\br{\hspace*{\fill} \\* }
    % Math Jax compatibility definitions
    \def\gt{>}
    \def\lt{<}
    \let\Oldtex\TeX
    \let\Oldlatex\LaTeX
    \renewcommand{\TeX}{\textrm{\Oldtex}}
    \renewcommand{\LaTeX}{\textrm{\Oldlatex}}
    % Document parameters
    % Document title
    \title{LogDog}
    
    
    
    
    
% Pygments definitions
\makeatletter
\def\PY@reset{\let\PY@it=\relax \let\PY@bf=\relax%
    \let\PY@ul=\relax \let\PY@tc=\relax%
    \let\PY@bc=\relax \let\PY@ff=\relax}
\def\PY@tok#1{\csname PY@tok@#1\endcsname}
\def\PY@toks#1+{\ifx\relax#1\empty\else%
    \PY@tok{#1}\expandafter\PY@toks\fi}
\def\PY@do#1{\PY@bc{\PY@tc{\PY@ul{%
    \PY@it{\PY@bf{\PY@ff{#1}}}}}}}
\def\PY#1#2{\PY@reset\PY@toks#1+\relax+\PY@do{#2}}

\@namedef{PY@tok@w}{\def\PY@tc##1{\textcolor[rgb]{0.73,0.73,0.73}{##1}}}
\@namedef{PY@tok@c}{\let\PY@it=\textit\def\PY@tc##1{\textcolor[rgb]{0.25,0.50,0.50}{##1}}}
\@namedef{PY@tok@cp}{\def\PY@tc##1{\textcolor[rgb]{0.74,0.48,0.00}{##1}}}
\@namedef{PY@tok@k}{\let\PY@bf=\textbf\def\PY@tc##1{\textcolor[rgb]{0.00,0.50,0.00}{##1}}}
\@namedef{PY@tok@kp}{\def\PY@tc##1{\textcolor[rgb]{0.00,0.50,0.00}{##1}}}
\@namedef{PY@tok@kt}{\def\PY@tc##1{\textcolor[rgb]{0.69,0.00,0.25}{##1}}}
\@namedef{PY@tok@o}{\def\PY@tc##1{\textcolor[rgb]{0.40,0.40,0.40}{##1}}}
\@namedef{PY@tok@ow}{\let\PY@bf=\textbf\def\PY@tc##1{\textcolor[rgb]{0.67,0.13,1.00}{##1}}}
\@namedef{PY@tok@nb}{\def\PY@tc##1{\textcolor[rgb]{0.00,0.50,0.00}{##1}}}
\@namedef{PY@tok@nf}{\def\PY@tc##1{\textcolor[rgb]{0.00,0.00,1.00}{##1}}}
\@namedef{PY@tok@nc}{\let\PY@bf=\textbf\def\PY@tc##1{\textcolor[rgb]{0.00,0.00,1.00}{##1}}}
\@namedef{PY@tok@nn}{\let\PY@bf=\textbf\def\PY@tc##1{\textcolor[rgb]{0.00,0.00,1.00}{##1}}}
\@namedef{PY@tok@ne}{\let\PY@bf=\textbf\def\PY@tc##1{\textcolor[rgb]{0.82,0.25,0.23}{##1}}}
\@namedef{PY@tok@nv}{\def\PY@tc##1{\textcolor[rgb]{0.10,0.09,0.49}{##1}}}
\@namedef{PY@tok@no}{\def\PY@tc##1{\textcolor[rgb]{0.53,0.00,0.00}{##1}}}
\@namedef{PY@tok@nl}{\def\PY@tc##1{\textcolor[rgb]{0.63,0.63,0.00}{##1}}}
\@namedef{PY@tok@ni}{\let\PY@bf=\textbf\def\PY@tc##1{\textcolor[rgb]{0.60,0.60,0.60}{##1}}}
\@namedef{PY@tok@na}{\def\PY@tc##1{\textcolor[rgb]{0.49,0.56,0.16}{##1}}}
\@namedef{PY@tok@nt}{\let\PY@bf=\textbf\def\PY@tc##1{\textcolor[rgb]{0.00,0.50,0.00}{##1}}}
\@namedef{PY@tok@nd}{\def\PY@tc##1{\textcolor[rgb]{0.67,0.13,1.00}{##1}}}
\@namedef{PY@tok@s}{\def\PY@tc##1{\textcolor[rgb]{0.73,0.13,0.13}{##1}}}
\@namedef{PY@tok@sd}{\let\PY@it=\textit\def\PY@tc##1{\textcolor[rgb]{0.73,0.13,0.13}{##1}}}
\@namedef{PY@tok@si}{\let\PY@bf=\textbf\def\PY@tc##1{\textcolor[rgb]{0.73,0.40,0.53}{##1}}}
\@namedef{PY@tok@se}{\let\PY@bf=\textbf\def\PY@tc##1{\textcolor[rgb]{0.73,0.40,0.13}{##1}}}
\@namedef{PY@tok@sr}{\def\PY@tc##1{\textcolor[rgb]{0.73,0.40,0.53}{##1}}}
\@namedef{PY@tok@ss}{\def\PY@tc##1{\textcolor[rgb]{0.10,0.09,0.49}{##1}}}
\@namedef{PY@tok@sx}{\def\PY@tc##1{\textcolor[rgb]{0.00,0.50,0.00}{##1}}}
\@namedef{PY@tok@m}{\def\PY@tc##1{\textcolor[rgb]{0.40,0.40,0.40}{##1}}}
\@namedef{PY@tok@gh}{\let\PY@bf=\textbf\def\PY@tc##1{\textcolor[rgb]{0.00,0.00,0.50}{##1}}}
\@namedef{PY@tok@gu}{\let\PY@bf=\textbf\def\PY@tc##1{\textcolor[rgb]{0.50,0.00,0.50}{##1}}}
\@namedef{PY@tok@gd}{\def\PY@tc##1{\textcolor[rgb]{0.63,0.00,0.00}{##1}}}
\@namedef{PY@tok@gi}{\def\PY@tc##1{\textcolor[rgb]{0.00,0.63,0.00}{##1}}}
\@namedef{PY@tok@gr}{\def\PY@tc##1{\textcolor[rgb]{1.00,0.00,0.00}{##1}}}
\@namedef{PY@tok@ge}{\let\PY@it=\textit}
\@namedef{PY@tok@gs}{\let\PY@bf=\textbf}
\@namedef{PY@tok@gp}{\let\PY@bf=\textbf\def\PY@tc##1{\textcolor[rgb]{0.00,0.00,0.50}{##1}}}
\@namedef{PY@tok@go}{\def\PY@tc##1{\textcolor[rgb]{0.53,0.53,0.53}{##1}}}
\@namedef{PY@tok@gt}{\def\PY@tc##1{\textcolor[rgb]{0.00,0.27,0.87}{##1}}}
\@namedef{PY@tok@err}{\def\PY@bc##1{{\setlength{\fboxsep}{\string -\fboxrule}\fcolorbox[rgb]{1.00,0.00,0.00}{1,1,1}{\strut ##1}}}}
\@namedef{PY@tok@kc}{\let\PY@bf=\textbf\def\PY@tc##1{\textcolor[rgb]{0.00,0.50,0.00}{##1}}}
\@namedef{PY@tok@kd}{\let\PY@bf=\textbf\def\PY@tc##1{\textcolor[rgb]{0.00,0.50,0.00}{##1}}}
\@namedef{PY@tok@kn}{\let\PY@bf=\textbf\def\PY@tc##1{\textcolor[rgb]{0.00,0.50,0.00}{##1}}}
\@namedef{PY@tok@kr}{\let\PY@bf=\textbf\def\PY@tc##1{\textcolor[rgb]{0.00,0.50,0.00}{##1}}}
\@namedef{PY@tok@bp}{\def\PY@tc##1{\textcolor[rgb]{0.00,0.50,0.00}{##1}}}
\@namedef{PY@tok@fm}{\def\PY@tc##1{\textcolor[rgb]{0.00,0.00,1.00}{##1}}}
\@namedef{PY@tok@vc}{\def\PY@tc##1{\textcolor[rgb]{0.10,0.09,0.49}{##1}}}
\@namedef{PY@tok@vg}{\def\PY@tc##1{\textcolor[rgb]{0.10,0.09,0.49}{##1}}}
\@namedef{PY@tok@vi}{\def\PY@tc##1{\textcolor[rgb]{0.10,0.09,0.49}{##1}}}
\@namedef{PY@tok@vm}{\def\PY@tc##1{\textcolor[rgb]{0.10,0.09,0.49}{##1}}}
\@namedef{PY@tok@sa}{\def\PY@tc##1{\textcolor[rgb]{0.73,0.13,0.13}{##1}}}
\@namedef{PY@tok@sb}{\def\PY@tc##1{\textcolor[rgb]{0.73,0.13,0.13}{##1}}}
\@namedef{PY@tok@sc}{\def\PY@tc##1{\textcolor[rgb]{0.73,0.13,0.13}{##1}}}
\@namedef{PY@tok@dl}{\def\PY@tc##1{\textcolor[rgb]{0.73,0.13,0.13}{##1}}}
\@namedef{PY@tok@s2}{\def\PY@tc##1{\textcolor[rgb]{0.73,0.13,0.13}{##1}}}
\@namedef{PY@tok@sh}{\def\PY@tc##1{\textcolor[rgb]{0.73,0.13,0.13}{##1}}}
\@namedef{PY@tok@s1}{\def\PY@tc##1{\textcolor[rgb]{0.73,0.13,0.13}{##1}}}
\@namedef{PY@tok@mb}{\def\PY@tc##1{\textcolor[rgb]{0.40,0.40,0.40}{##1}}}
\@namedef{PY@tok@mf}{\def\PY@tc##1{\textcolor[rgb]{0.40,0.40,0.40}{##1}}}
\@namedef{PY@tok@mh}{\def\PY@tc##1{\textcolor[rgb]{0.40,0.40,0.40}{##1}}}
\@namedef{PY@tok@mi}{\def\PY@tc##1{\textcolor[rgb]{0.40,0.40,0.40}{##1}}}
\@namedef{PY@tok@il}{\def\PY@tc##1{\textcolor[rgb]{0.40,0.40,0.40}{##1}}}
\@namedef{PY@tok@mo}{\def\PY@tc##1{\textcolor[rgb]{0.40,0.40,0.40}{##1}}}
\@namedef{PY@tok@ch}{\let\PY@it=\textit\def\PY@tc##1{\textcolor[rgb]{0.25,0.50,0.50}{##1}}}
\@namedef{PY@tok@cm}{\let\PY@it=\textit\def\PY@tc##1{\textcolor[rgb]{0.25,0.50,0.50}{##1}}}
\@namedef{PY@tok@cpf}{\let\PY@it=\textit\def\PY@tc##1{\textcolor[rgb]{0.25,0.50,0.50}{##1}}}
\@namedef{PY@tok@c1}{\let\PY@it=\textit\def\PY@tc##1{\textcolor[rgb]{0.25,0.50,0.50}{##1}}}
\@namedef{PY@tok@cs}{\let\PY@it=\textit\def\PY@tc##1{\textcolor[rgb]{0.25,0.50,0.50}{##1}}}

\def\PYZbs{\char`\\}
\def\PYZus{\char`\_}
\def\PYZob{\char`\{}
\def\PYZcb{\char`\}}
\def\PYZca{\char`\^}
\def\PYZam{\char`\&}
\def\PYZlt{\char`\<}
\def\PYZgt{\char`\>}
\def\PYZsh{\char`\#}
\def\PYZpc{\char`\%}
\def\PYZdl{\char`\$}
\def\PYZhy{\char`\-}
\def\PYZsq{\char`\'}
\def\PYZdq{\char`\"}
\def\PYZti{\char`\~}
% for compatibility with earlier versions
\def\PYZat{@}
\def\PYZlb{[}
\def\PYZrb{]}
\makeatother


    % For linebreaks inside Verbatim environment from package fancyvrb. 
    \makeatletter
        \newbox\Wrappedcontinuationbox 
        \newbox\Wrappedvisiblespacebox 
        \newcommand*\Wrappedvisiblespace {\textcolor{red}{\textvisiblespace}} 
        \newcommand*\Wrappedcontinuationsymbol {\textcolor{red}{\llap{\tiny$\m@th\hookrightarrow$}}} 
        \newcommand*\Wrappedcontinuationindent {3ex } 
        \newcommand*\Wrappedafterbreak {\kern\Wrappedcontinuationindent\copy\Wrappedcontinuationbox} 
        % Take advantage of the already applied Pygments mark-up to insert 
        % potential linebreaks for TeX processing. 
        %        {, <, #, %, $, ' and ": go to next line. 
        %        _, }, ^, &, >, - and ~: stay at end of broken line. 
        % Use of \textquotesingle for straight quote. 
        \newcommand*\Wrappedbreaksatspecials {% 
            \def\PYGZus{\discretionary{\char`\_}{\Wrappedafterbreak}{\char`\_}}% 
            \def\PYGZob{\discretionary{}{\Wrappedafterbreak\char`\{}{\char`\{}}% 
            \def\PYGZcb{\discretionary{\char`\}}{\Wrappedafterbreak}{\char`\}}}% 
            \def\PYGZca{\discretionary{\char`\^}{\Wrappedafterbreak}{\char`\^}}% 
            \def\PYGZam{\discretionary{\char`\&}{\Wrappedafterbreak}{\char`\&}}% 
            \def\PYGZlt{\discretionary{}{\Wrappedafterbreak\char`\<}{\char`\<}}% 
            \def\PYGZgt{\discretionary{\char`\>}{\Wrappedafterbreak}{\char`\>}}% 
            \def\PYGZsh{\discretionary{}{\Wrappedafterbreak\char`\#}{\char`\#}}% 
            \def\PYGZpc{\discretionary{}{\Wrappedafterbreak\char`\%}{\char`\%}}% 
            \def\PYGZdl{\discretionary{}{\Wrappedafterbreak\char`\$}{\char`\$}}% 
            \def\PYGZhy{\discretionary{\char`\-}{\Wrappedafterbreak}{\char`\-}}% 
            \def\PYGZsq{\discretionary{}{\Wrappedafterbreak\textquotesingle}{\textquotesingle}}% 
            \def\PYGZdq{\discretionary{}{\Wrappedafterbreak\char`\"}{\char`\"}}% 
            \def\PYGZti{\discretionary{\char`\~}{\Wrappedafterbreak}{\char`\~}}% 
        } 
        % Some characters . , ; ? ! / are not pygmentized. 
        % This macro makes them "active" and they will insert potential linebreaks 
        \newcommand*\Wrappedbreaksatpunct {% 
            \lccode`\~`\.\lowercase{\def~}{\discretionary{\hbox{\char`\.}}{\Wrappedafterbreak}{\hbox{\char`\.}}}% 
            \lccode`\~`\,\lowercase{\def~}{\discretionary{\hbox{\char`\,}}{\Wrappedafterbreak}{\hbox{\char`\,}}}% 
            \lccode`\~`\;\lowercase{\def~}{\discretionary{\hbox{\char`\;}}{\Wrappedafterbreak}{\hbox{\char`\;}}}% 
            \lccode`\~`\:\lowercase{\def~}{\discretionary{\hbox{\char`\:}}{\Wrappedafterbreak}{\hbox{\char`\:}}}% 
            \lccode`\~`\?\lowercase{\def~}{\discretionary{\hbox{\char`\?}}{\Wrappedafterbreak}{\hbox{\char`\?}}}% 
            \lccode`\~`\!\lowercase{\def~}{\discretionary{\hbox{\char`\!}}{\Wrappedafterbreak}{\hbox{\char`\!}}}% 
            \lccode`\~`\/\lowercase{\def~}{\discretionary{\hbox{\char`\/}}{\Wrappedafterbreak}{\hbox{\char`\/}}}% 
            \catcode`\.\active
            \catcode`\,\active 
            \catcode`\;\active
            \catcode`\:\active
            \catcode`\?\active
            \catcode`\!\active
            \catcode`\/\active 
            \lccode`\~`\~ 	
        }
    \makeatother

    \let\OriginalVerbatim=\Verbatim
    \makeatletter
    \renewcommand{\Verbatim}[1][1]{%
        %\parskip\z@skip
        \sbox\Wrappedcontinuationbox {\Wrappedcontinuationsymbol}%
        \sbox\Wrappedvisiblespacebox {\FV@SetupFont\Wrappedvisiblespace}%
        \def\FancyVerbFormatLine ##1{\hsize\linewidth
            \vtop{\raggedright\hyphenpenalty\z@\exhyphenpenalty\z@
                \doublehyphendemerits\z@\finalhyphendemerits\z@
                \strut ##1\strut}%
        }%
        % If the linebreak is at a space, the latter will be displayed as visible
        % space at end of first line, and a continuation symbol starts next line.
        % Stretch/shrink are however usually zero for typewriter font.
        \def\FV@Space {%
            \nobreak\hskip\z@ plus\fontdimen3\font minus\fontdimen4\font
            \discretionary{\copy\Wrappedvisiblespacebox}{\Wrappedafterbreak}
            {\kern\fontdimen2\font}%
        }%
        
        % Allow breaks at special characters using \PYG... macros.
        \Wrappedbreaksatspecials
        % Breaks at punctuation characters . , ; ? ! and / need catcode=\active 	
        \OriginalVerbatim[#1,codes*=\Wrappedbreaksatpunct]%
    }
    \makeatother

    % Exact colors from NB
    \definecolor{incolor}{HTML}{303F9F}
    \definecolor{outcolor}{HTML}{D84315}
    \definecolor{cellborder}{HTML}{CFCFCF}
    \definecolor{cellbackground}{HTML}{F7F7F7}
    
    % prompt
    \makeatletter
    \newcommand{\boxspacing}{\kern\kvtcb@left@rule\kern\kvtcb@boxsep}
    \makeatother
    \newcommand{\prompt}[4]{
        {\ttfamily\llap{{\color{#2}[#3]:\hspace{3pt}#4}}\vspace{-\baselineskip}}
    }
    

    
    % Prevent overflowing lines due to hard-to-break entities
    \sloppy 
    % Setup hyperref package
    \hypersetup{
      breaklinks=true,  % so long urls are correctly broken across lines
      colorlinks=true,
      urlcolor=urlcolor,
      linkcolor=linkcolor,
      citecolor=citecolor,
      }
    % Slightly bigger margins than the latex defaults
    
    \geometry{verbose,tmargin=1in,bmargin=1in,lmargin=1in,rmargin=1in}
    
    

\begin{document}
    
    \maketitle
    
    

    
    \begin{tcolorbox}[breakable, size=fbox, boxrule=1pt, pad at break*=1mm,colback=cellbackground, colframe=cellborder]
\prompt{In}{incolor}{104}{\boxspacing}
\begin{Verbatim}[commandchars=\\\{\}]
\PY{c+c1}{\PYZsh{}CONVOLUTION에 후처리 추가.}

\PY{k+kn}{import} \PY{n+nn}{numpy} \PY{k}{as} \PY{n+nn}{np} 

\PY{k}{def} \PY{n+nf}{conv}\PY{p}{(}\PY{n}{src}\PY{p}{,}\PY{n}{kernel}\PY{p}{,}\PY{n}{stride}\PY{o}{=}\PY{l+m+mi}{1}\PY{p}{,}\PY{n}{zero\PYZus{}padding}\PY{o}{=}\PY{k+kc}{None}\PY{p}{)}\PY{p}{:}
    \PY{n}{result}\PY{o}{=}\PY{p}{[}\PY{p}{]}

    \PY{n}{f\PYZus{}y}\PY{p}{,} \PY{n}{f\PYZus{}x}\PY{o}{=} \PY{n}{np}\PY{o}{.}\PY{n}{shape}\PY{p}{(}\PY{n}{src}\PY{p}{)}\PY{c+c1}{\PYZsh{} 영상의 크기}
    \PY{n+nb}{print}\PY{p}{(}\PY{n}{f\PYZus{}y}\PY{p}{,}\PY{n}{f\PYZus{}x}\PY{p}{)}
    \PY{n}{k\PYZus{}y}\PY{p}{,} \PY{n}{k\PYZus{}x}\PY{o}{=} \PY{n}{np}\PY{o}{.}\PY{n}{shape}\PY{p}{(}\PY{n}{kernel}\PY{p}{)}\PY{c+c1}{\PYZsh{}커널의 크기}
    
    
    \PY{c+c1}{\PYZsh{}padding이 없다면, 자동으로 이미지 사이즈 유지하도록 설정.}
    \PY{k}{if} \PY{n}{zero\PYZus{}padding} \PY{o+ow}{is} \PY{k+kc}{None}\PY{p}{:}
        \PY{c+c1}{\PYZsh{}same padding 만들기위해 패딩 사이즈 계산.}
        \PY{n}{padding\PYZus{}y} \PY{o}{=} \PY{n+nb}{int}\PY{p}{(}\PY{p}{(}\PY{n}{stride} \PY{o}{*} \PY{n}{f\PYZus{}y}\PY{o}{\PYZhy{}}\PY{n}{stride}\PY{o}{\PYZhy{}}\PY{n}{f\PYZus{}y}\PY{o}{+}\PY{n}{k\PYZus{}y}\PY{p}{)}\PY{o}{/}\PY{l+m+mi}{2}\PY{p}{)}
        
        \PY{n}{padding\PYZus{}x} \PY{o}{=} \PY{n+nb}{int}\PY{p}{(}\PY{p}{(}\PY{n}{stride} \PY{o}{*} \PY{n}{f\PYZus{}x}\PY{o}{\PYZhy{}}\PY{n}{stride}\PY{o}{\PYZhy{}}\PY{n}{f\PYZus{}x}\PY{o}{+}\PY{n}{k\PYZus{}x}\PY{p}{)}\PY{o}{/}\PY{l+m+mi}{2}\PY{p}{)}
        \PY{n+nb}{print}\PY{p}{(}\PY{n}{padding\PYZus{}y}\PY{p}{,}\PY{n}{padding\PYZus{}x}\PY{p}{)}
        
        \PY{n}{src} \PY{o}{=} \PY{n}{np}\PY{o}{.}\PY{n}{pad}\PY{p}{(}\PY{n}{src}\PY{p}{,} \PY{p}{(}\PY{p}{(}\PY{n}{padding\PYZus{}y}\PY{p}{,}\PY{n}{padding\PYZus{}y}\PY{p}{)}\PY{p}{,}\PY{p}{(}\PY{n}{padding\PYZus{}x}\PY{p}{,}\PY{n}{padding\PYZus{}x}\PY{p}{)}\PY{p}{)}\PY{p}{,} \PY{l+s+s1}{\PYZsq{}}\PY{l+s+s1}{constant}\PY{l+s+s1}{\PYZsq{}}\PY{p}{,} \PY{n}{constant\PYZus{}values}\PY{o}{=}\PY{l+m+mi}{0}\PY{p}{)}
    \PY{k}{elif} \PY{n}{zero\PYZus{}padding}\PY{o}{!=}\PY{l+m+mi}{0}\PY{p}{:}
        \PY{c+c1}{\PYZsh{} 0을 제외하고 zero padding}
        \PY{n}{padding\PYZus{}y}\PY{o}{=}\PY{n}{zero\PYZus{}padding}
        \PY{n}{padding\PYZus{}x}\PY{o}{=}\PY{n}{zero\PYZus{}padding}
        \PY{n}{src} \PY{o}{=} \PY{n}{np}\PY{o}{.}\PY{n}{pad}\PY{p}{(}\PY{n}{src}\PY{p}{,} \PY{p}{(}\PY{p}{(}\PY{n}{zero\PYZus{}padding}\PY{p}{,}\PY{n}{zero\PYZus{}padding}\PY{p}{)}\PY{p}{,}\PY{p}{(}\PY{n}{zero\PYZus{}padding}\PY{p}{,}\PY{n}{zero\PYZus{}padding}\PY{p}{)}\PY{p}{)}\PY{p}{,} \PY{l+s+s1}{\PYZsq{}}\PY{l+s+s1}{constant}\PY{l+s+s1}{\PYZsq{}}\PY{p}{,} \PY{n}{constant\PYZus{}values}\PY{o}{=}\PY{l+m+mi}{0}\PY{p}{)}
    \PY{k}{else}\PY{p}{:}
        \PY{c+c1}{\PYZsh{}0일때, 패딩을 주지않는다.}
        \PY{n}{padding\PYZus{}y}\PY{o}{=}\PY{l+m+mi}{0}
        \PY{n}{padding\PYZus{}x}\PY{o}{=}\PY{l+m+mi}{0}
    \PY{n+nb}{print}\PY{p}{(}\PY{l+s+s2}{\PYZdq{}}\PY{l+s+s2}{입력 : }\PY{l+s+se}{\PYZbs{}n}\PY{l+s+s2}{\PYZdq{}}\PY{p}{,}\PY{n}{src}\PY{p}{)}
    
    \PY{c+c1}{\PYZsh{}이미지 사이즈 다시 계산}
    \PY{n}{f\PYZus{}y}\PY{p}{,} \PY{n}{f\PYZus{}x}\PY{o}{=} \PY{n}{np}\PY{o}{.}\PY{n}{shape}\PY{p}{(}\PY{n}{src}\PY{p}{)}\PY{c+c1}{\PYZsh{} 영상의 크기}
    \PY{c+c1}{\PYZsh{}print(f\PYZus{}y,f\PYZus{}x)}
    
    \PY{c+c1}{\PYZsh{}패딩이 생기면 시작점도 달라야한다.}
    \PY{n}{start\PYZus{}y} \PY{o}{=} \PY{n}{padding\PYZus{}y}
    \PY{n}{start\PYZus{}x} \PY{o}{=} \PY{n}{padding\PYZus{}x}
    
    \PY{k}{if} \PY{n}{zero\PYZus{}padding} \PY{o}{!=}\PY{l+m+mi}{0}\PY{p}{:}
        \PY{c+c1}{\PYZsh{}print(\PYZsq{}y 좌표 최대 \PYZsq{},f\PYZus{}y\PYZhy{}padding\PYZus{}y)}
        \PY{c+c1}{\PYZsh{}print(\PYZsq{}x 좌표 최대 \PYZsq{},f\PYZus{}x\PYZhy{}padding\PYZus{}x)}
        \PY{k}{for} \PY{n}{y} \PY{o+ow}{in} \PY{n+nb}{range}\PY{p}{(}\PY{n}{start\PYZus{}y}\PY{p}{,}\PY{n}{f\PYZus{}y}\PY{o}{\PYZhy{}}\PY{n}{padding\PYZus{}y}\PY{p}{,}\PY{n}{stride}\PY{p}{)}\PY{p}{:}
            \PY{k}{for} \PY{n}{x} \PY{o+ow}{in} \PY{n+nb}{range}\PY{p}{(}\PY{n}{start\PYZus{}x}\PY{p}{,}\PY{n}{f\PYZus{}x}\PY{o}{\PYZhy{}}\PY{n}{padding\PYZus{}x}\PY{p}{,}\PY{n}{stride}\PY{p}{)}\PY{p}{:}
                \PY{n}{result}\PY{o}{.}\PY{n}{append}\PY{p}{(}\PY{p}{(}\PY{n}{src}\PY{p}{[}\PY{n}{y}\PY{o}{\PYZhy{}}\PY{n}{padding\PYZus{}y}\PY{p}{:}\PY{n}{y}\PY{o}{+}\PY{n}{k\PYZus{}y}\PY{o}{\PYZhy{}}\PY{n}{padding\PYZus{}y}\PY{p}{,} \PY{n}{x}\PY{o}{\PYZhy{}}\PY{n}{padding\PYZus{}x}\PY{p}{:}\PY{n}{x}\PY{o}{+}\PY{n}{k\PYZus{}x}\PY{o}{\PYZhy{}}\PY{n}{padding\PYZus{}x}\PY{p}{]}\PY{o}{*} \PY{n}{kernel}\PY{p}{)}\PY{o}{.}\PY{n}{sum}\PY{p}{(}\PY{p}{)}\PY{p}{)}
        \PY{c+c1}{\PYZsh{}영상 사이즈는 이전 사이즈 그대로.}
        \PY{n}{OUTPUT\PYZus{}Y}\PY{o}{=}\PY{n}{f\PYZus{}y} \PY{o}{\PYZhy{}} \PY{l+m+mi}{2}\PY{o}{*}\PY{n}{padding\PYZus{}y}
        \PY{n}{OUTPUT\PYZus{}X}\PY{o}{=}\PY{n}{f\PYZus{}x} \PY{o}{\PYZhy{}} \PY{l+m+mi}{2}\PY{o}{*}\PY{n}{padding\PYZus{}x}
                
                
    \PY{k}{else}\PY{p}{:}
        \PY{c+c1}{\PYZsh{}print(\PYZsq{}y 좌표 최대 \PYZsq{},f\PYZus{}y\PYZhy{}k\PYZus{}y+1)}
        \PY{c+c1}{\PYZsh{}print(\PYZsq{}x 좌표 최대 \PYZsq{},f\PYZus{}x\PYZhy{}k\PYZus{}x+1)}
        \PY{k}{for} \PY{n}{y} \PY{o+ow}{in} \PY{n+nb}{range}\PY{p}{(}\PY{n}{start\PYZus{}y}\PY{p}{,}\PY{n}{f\PYZus{}y}\PY{o}{\PYZhy{}}\PY{n}{k\PYZus{}y}\PY{o}{+}\PY{l+m+mi}{1}\PY{p}{,}\PY{n}{stride}\PY{p}{)}\PY{p}{:}
            \PY{k}{for} \PY{n}{x} \PY{o+ow}{in} \PY{n+nb}{range}\PY{p}{(}\PY{n}{start\PYZus{}x}\PY{p}{,}\PY{n}{f\PYZus{}x}\PY{o}{\PYZhy{}}\PY{n}{k\PYZus{}x}\PY{o}{+}\PY{l+m+mi}{1}\PY{p}{,}\PY{n}{stride}\PY{p}{)}\PY{p}{:}
                \PY{n}{result}\PY{o}{.}\PY{n}{append}\PY{p}{(}\PY{p}{(}\PY{n}{src}\PY{p}{[}\PY{n}{y}\PY{p}{:}\PY{n}{y}\PY{o}{+}\PY{n}{k\PYZus{}y}\PY{p}{,}\PY{n}{x} \PY{p}{:} \PY{n}{x}\PY{o}{+}\PY{n}{k\PYZus{}x}\PY{p}{]}\PY{o}{*} \PY{n}{kernel}\PY{p}{)}\PY{o}{.}\PY{n}{sum}\PY{p}{(}\PY{p}{)}\PY{p}{)}
        \PY{c+c1}{\PYZsh{}영상 크기 계산}
        \PY{n}{OUTPUT\PYZus{}Y}\PY{o}{=}\PY{n+nb}{int}\PY{p}{(} \PY{p}{(}\PY{p}{(}\PY{n}{f\PYZus{}y}\PY{o}{\PYZhy{}}\PY{n}{k\PYZus{}y}\PY{p}{)}\PY{o}{/}\PY{n}{stride}\PY{p}{)}\PY{o}{+}\PY{l+m+mi}{1} \PY{p}{)}
        \PY{n}{OUTPUT\PYZus{}X}\PY{o}{=}\PY{n+nb}{int}\PY{p}{(} \PY{p}{(}\PY{p}{(}\PY{n}{f\PYZus{}x}\PY{o}{\PYZhy{}}\PY{n}{k\PYZus{}x}\PY{p}{)}\PY{o}{/}\PY{n}{stride}\PY{p}{)}\PY{o}{+}\PY{l+m+mi}{1} \PY{p}{)}
        \PY{c+c1}{\PYZsh{}print(OUTPUT\PYZus{}Y,OUTPUT\PYZus{}X)}
    
    
        
    \PY{n}{result}\PY{o}{=}\PY{n}{np}\PY{o}{.}\PY{n}{array}\PY{p}{(}\PY{n}{result}\PY{p}{)}\PY{o}{.}\PY{n}{reshape}\PY{p}{(}\PY{n}{OUTPUT\PYZus{}Y}\PY{p}{,}\PY{n}{OUTPUT\PYZus{}X}\PY{p}{)}
    \PY{c+c1}{\PYZsh{}후처리( 0보다 작으면 0, 255보다 크면 255)}
    \PY{n}{result}\PY{p}{[}\PY{n}{result}\PY{o}{\PYZlt{}}\PY{l+m+mi}{0}\PY{p}{]}\PY{o}{=}\PY{l+m+mi}{0}
    \PY{n}{result}\PY{p}{[}\PY{n}{result}\PY{o}{\PYZgt{}}\PY{l+m+mi}{255}\PY{p}{]}\PY{o}{=}\PY{l+m+mi}{255}
    \PY{k}{return} \PY{n}{result}
    
    
\end{Verbatim}
\end{tcolorbox}

    \begin{tcolorbox}[breakable, size=fbox, boxrule=1pt, pad at break*=1mm,colback=cellbackground, colframe=cellborder]
\prompt{In}{incolor}{97}{\boxspacing}
\begin{Verbatim}[commandchars=\\\{\}]
\PY{c+c1}{\PYZsh{} EX) 소벨 마스크를 이용한 에지 검출. 에지와 관련된 맵들.}
\PY{k+kn}{import} \PY{n+nn}{cv2}
\PY{k+kn}{import} \PY{n+nn}{numpy} \PY{k}{as} \PY{n+nn}{np}
\PY{k+kn}{from} \PY{n+nn}{matplotlib} \PY{k+kn}{import} \PY{n}{pyplot} \PY{k}{as} \PY{n}{plt}

\PY{n}{img} \PY{o}{=} \PY{n}{cv2}\PY{o}{.}\PY{n}{imread}\PY{p}{(}\PY{l+s+s2}{\PYZdq{}}\PY{l+s+s2}{../../data/find\PYZus{}face.jpg}\PY{l+s+s2}{\PYZdq{}}\PY{p}{,}\PY{l+m+mi}{0}\PY{p}{)} \PY{c+c1}{\PYZsh{} gray scale로 읽어서, 1 채널 명암값에 필터를 적용한다.}

\PY{c+c1}{\PYZsh{} 소벨 커널을 직접 생성해서 엣지 검출}
\PY{c+c1}{\PYZsh{}\PYZsh{} 소벨 커널 생성}
\PY{n}{gx\PYZus{}k} \PY{o}{=} \PY{n}{np}\PY{o}{.}\PY{n}{array}\PY{p}{(}\PY{p}{[}\PY{p}{[}\PY{o}{\PYZhy{}}\PY{l+m+mi}{1}\PY{p}{,}\PY{l+m+mi}{0}\PY{p}{,}\PY{l+m+mi}{1}\PY{p}{]}\PY{p}{,} \PY{p}{[}\PY{o}{\PYZhy{}}\PY{l+m+mi}{2}\PY{p}{,}\PY{l+m+mi}{0}\PY{p}{,}\PY{l+m+mi}{2}\PY{p}{]}\PY{p}{,}\PY{p}{[}\PY{o}{\PYZhy{}}\PY{l+m+mi}{1}\PY{p}{,}\PY{l+m+mi}{0}\PY{p}{,}\PY{l+m+mi}{1}\PY{p}{]}\PY{p}{]}\PY{p}{)}

\PY{n}{edge\PYZus{}dx} \PY{o}{=} \PY{n}{cv2}\PY{o}{.}\PY{n}{filter2D}\PY{p}{(}\PY{n}{img}\PY{p}{,} \PY{o}{\PYZhy{}}\PY{l+m+mi}{1}\PY{p}{,} \PY{n}{gx\PYZus{}k}\PY{p}{,}\PY{n}{borderType}\PY{o}{=}\PY{l+m+mi}{0}\PY{p}{)}
\PY{c+c1}{\PYZsh{}print(\PYZdq{}filter 2d 결과 \PYZbs{}n\PYZdq{},edge\PYZus{}dx)}
\PY{n}{edge\PYZus{}dx\PYZus{}self} \PY{o}{=} \PY{n}{conv}\PY{p}{(}\PY{n}{src}\PY{o}{=}\PY{n}{img}\PY{p}{,} \PY{n}{kernel}\PY{o}{=}\PY{n}{gx\PYZus{}k}\PY{p}{)}
\PY{c+c1}{\PYZsh{}print(\PYZdq{}결과 \PYZbs{}n\PYZdq{},edge\PYZus{}dx\PYZus{}self)}


\PY{n}{fig} \PY{o}{=}\PY{n}{plt}\PY{o}{.}\PY{n}{figure}\PY{p}{(}\PY{n}{figsize}\PY{o}{=}\PY{p}{(}\PY{l+m+mi}{10}\PY{p}{,}\PY{l+m+mi}{10}\PY{p}{)}\PY{p}{)}

\PY{n}{plt}\PY{o}{.}\PY{n}{subplot}\PY{p}{(}\PY{l+m+mi}{221}\PY{p}{)}\PY{c+c1}{\PYZsh{} 1행 2열중 1번째}
\PY{n}{plt}\PY{o}{.}\PY{n}{imshow}\PY{p}{(}\PY{n}{img}\PY{p}{,}\PY{l+s+s1}{\PYZsq{}}\PY{l+s+s1}{gray}\PY{l+s+s1}{\PYZsq{}}\PY{p}{)}
\PY{n}{plt}\PY{o}{.}\PY{n}{title}\PY{p}{(}\PY{l+s+s1}{\PYZsq{}}\PY{l+s+s1}{filter2D}\PY{l+s+s1}{\PYZsq{}}\PY{p}{)}

\PY{n}{plt}\PY{o}{.}\PY{n}{subplot}\PY{p}{(}\PY{l+m+mi}{223}\PY{p}{)}\PY{c+c1}{\PYZsh{} 1행 2열중 1번째}
\PY{n}{plt}\PY{o}{.}\PY{n}{imshow}\PY{p}{(}\PY{n}{edge\PYZus{}dx}\PY{p}{,}\PY{l+s+s1}{\PYZsq{}}\PY{l+s+s1}{gray}\PY{l+s+s1}{\PYZsq{}}\PY{p}{)}
\PY{n}{plt}\PY{o}{.}\PY{n}{title}\PY{p}{(}\PY{l+s+s1}{\PYZsq{}}\PY{l+s+s1}{filter2D}\PY{l+s+s1}{\PYZsq{}}\PY{p}{)}


\PY{n}{plt}\PY{o}{.}\PY{n}{subplot}\PY{p}{(}\PY{l+m+mi}{224}\PY{p}{)}\PY{c+c1}{\PYZsh{} 1행 2열중 1번째}
\PY{n}{plt}\PY{o}{.}\PY{n}{imshow}\PY{p}{(}\PY{n}{edge\PYZus{}dx\PYZus{}self}\PY{p}{,}\PY{l+s+s1}{\PYZsq{}}\PY{l+s+s1}{gray}\PY{l+s+s1}{\PYZsq{}}\PY{p}{)}
\PY{n}{plt}\PY{o}{.}\PY{n}{title}\PY{p}{(}\PY{l+s+s1}{\PYZsq{}}\PY{l+s+s1}{self}\PY{l+s+s1}{\PYZsq{}}\PY{p}{)}

\PY{c+c1}{\PYZsh{}결과가 다른 이유는 최대값과 음수에대한 예외처리로 발생.}
\PY{c+c1}{\PYZsh{}255이상은 255로, 음수는 전부 0}
\end{Verbatim}
\end{tcolorbox}

    \begin{Verbatim}[commandchars=\\\{\}]
305 458
1 1
입력 :
 [[  0   0   0 {\ldots}   0   0   0]
 [  0 234 234 {\ldots} 254 253   0]
 [  0 234 234 {\ldots} 254 253   0]
 {\ldots}
 [  0 210 209 {\ldots} 236 236   0]
 [  0 209 210 {\ldots} 235 235   0]
 [  0   0   0 {\ldots}   0   0   0]]
307 460
y 좌표 최대  306
x 좌표 최대  459
    \end{Verbatim}

            \begin{tcolorbox}[breakable, size=fbox, boxrule=.5pt, pad at break*=1mm, opacityfill=0]
\prompt{Out}{outcolor}{97}{\boxspacing}
\begin{Verbatim}[commandchars=\\\{\}]
Text(0.5, 1.0, 'self')
\end{Verbatim}
\end{tcolorbox}
        
    \begin{center}
    \adjustimage{max size={0.9\linewidth}{0.9\paperheight}}{LogDog_files/LogDog_1_2.png}
    \end{center}
    { \hspace*{\fill} \\}
    
    \hypertarget{uxac00uxc6b0uxc2dcuxc548-uxcee4uxb110-uxad6cuxd604}{%
\subsection{- 가우시안 커널
구현}\label{uxac00uxc6b0uxc2dcuxc548-uxcee4uxb110-uxad6cuxd604}}

마지막으로 합으로 나눠주면 된다.

커널의 한변 길이는 자동으로 (6시그마) or (6시그마+1)로 설정한다.

\begin{center}\rule{0.5\linewidth}{0.5pt}\end{center}

\begin{verbatim}
gaussian(sigma)

    - sigma : 표준편차를 넣으면 자동 생성.
\end{verbatim}

    \begin{tcolorbox}[breakable, size=fbox, boxrule=1pt, pad at break*=1mm,colback=cellbackground, colframe=cellborder]
\prompt{In}{incolor}{99}{\boxspacing}
\begin{Verbatim}[commandchars=\\\{\}]
\PY{c+c1}{\PYZsh{}ex) 가우시안 필터구현}
\PY{c+c1}{\PYZsh{}https://throwexception.tistory.com/1060}
\PY{k+kn}{import} \PY{n+nn}{cv2}
\PY{k+kn}{from} \PY{n+nn}{matplotlib} \PY{k+kn}{import} \PY{n}{pyplot} \PY{k}{as} \PY{n}{plt}

\PY{k}{def} \PY{n+nf}{gaussian}\PY{p}{(}\PY{n}{sigma}\PY{p}{)}\PY{p}{:}
    \PY{c+c1}{\PYZsh{}커널 사이즈 할당}
    \PY{n}{kernel\PYZus{}size} \PY{o}{=} \PY{n}{np}\PY{o}{.}\PY{n}{around}\PY{p}{(}\PY{l+m+mi}{6}\PY{o}{*}\PY{n}{sigma}\PY{p}{)}
    \PY{k}{if} \PY{n}{kernel\PYZus{}size}\PY{o}{\PYZpc{}}\PY{k}{2} ==0:
        \PY{n}{kernel\PYZus{}size}\PY{o}{=}\PY{n}{kernel\PYZus{}size}\PY{o}{+}\PY{l+m+mi}{1}
    \PY{n+nb}{print}\PY{p}{(}\PY{l+s+s2}{\PYZdq{}}\PY{l+s+s2}{커널 사이즈: }\PY{l+s+s2}{\PYZdq{}}\PY{p}{,}\PY{n}{kernel\PYZus{}size}\PY{p}{,}\PY{n}{kernel\PYZus{}size}\PY{p}{)}
    \PY{n}{a} \PY{o}{=} \PY{n}{kernel\PYZus{}size}\PY{o}{/}\PY{o}{/}\PY{l+m+mi}{2}
    
    \PY{c+c1}{\PYZsh{}x,y좌표 생성}
    \PY{n}{x}\PY{p}{,} \PY{n}{y} \PY{o}{=} \PY{n}{np}\PY{o}{.}\PY{n}{ogrid}\PY{p}{[}\PY{o}{\PYZhy{}}\PY{n}{a}\PY{p}{:}\PY{n}{a}\PY{o}{+}\PY{l+m+mi}{1}\PY{p}{,} \PY{o}{\PYZhy{}}\PY{n}{a}\PY{p}{:}\PY{n}{a}\PY{o}{+}\PY{l+m+mi}{1}\PY{p}{]}
    \PY{c+c1}{\PYZsh{}print(x,y)}
    
    \PY{c+c1}{\PYZsh{}가우시안 식 대입.}
    \PY{n}{kernel} \PY{o}{=} \PY{n}{np}\PY{o}{.}\PY{n}{exp}\PY{p}{(}\PY{o}{\PYZhy{}}\PY{p}{(}\PY{n}{y}\PY{o}{*}\PY{o}{*}\PY{l+m+mi}{2}\PY{o}{+}\PY{n}{x}\PY{o}{*}\PY{o}{*}\PY{l+m+mi}{2}\PY{p}{)}\PY{o}{/}\PY{p}{(}\PY{l+m+mi}{2}\PY{o}{*}\PY{n}{sigma}\PY{o}{*}\PY{o}{*}\PY{l+m+mi}{2}\PY{p}{)}\PY{p}{)}\PY{c+c1}{\PYZsh{} 상수는 제외}
    \PY{n}{kernel} \PY{o}{=} \PY{n}{kernel}\PY{o}{/}\PY{n}{np}\PY{o}{.}\PY{n}{sum}\PY{p}{(}\PY{n}{kernel}\PY{p}{)}
    \PY{c+c1}{\PYZsh{}print(\PYZsq{}가우시안 커널 \PYZbs{}n\PYZsq{},kernel)}
    
    \PY{k}{return} \PY{n}{kernel}
\end{Verbatim}
\end{tcolorbox}

    \begin{tcolorbox}[breakable, size=fbox, boxrule=1pt, pad at break*=1mm,colback=cellbackground, colframe=cellborder]
\prompt{In}{incolor}{67}{\boxspacing}
\begin{Verbatim}[commandchars=\\\{\}]
\PY{n}{gaussian}\PY{p}{(}\PY{l+m+mi}{1}\PY{o}{/}\PY{l+m+mi}{2}\PY{p}{)}
\end{Verbatim}
\end{tcolorbox}

    \begin{Verbatim}[commandchars=\\\{\}]
커널 사이즈:  3.0 3.0
가우시안 커널
 [[0.01134374 0.08381951 0.01134374]
 [0.08381951 0.61934703 0.08381951]
 [0.01134374 0.08381951 0.01134374]]
    \end{Verbatim}

            \begin{tcolorbox}[breakable, size=fbox, boxrule=.5pt, pad at break*=1mm, opacityfill=0]
\prompt{Out}{outcolor}{67}{\boxspacing}
\begin{Verbatim}[commandchars=\\\{\}]
array([[0.01134374, 0.08381951, 0.01134374],
       [0.08381951, 0.61934703, 0.08381951],
       [0.01134374, 0.08381951, 0.01134374]])
\end{Verbatim}
\end{tcolorbox}
        
    \begin{tcolorbox}[breakable, size=fbox, boxrule=1pt, pad at break*=1mm,colback=cellbackground, colframe=cellborder]
\prompt{In}{incolor}{112}{\boxspacing}
\begin{Verbatim}[commandchars=\\\{\}]
\PY{c+c1}{\PYZsh{} EX) 가우시안 구현. conv함수, 가우시안, filter2d 비교}
\PY{k+kn}{from} \PY{n+nn}{matplotlib} \PY{k+kn}{import} \PY{n}{font\PYZus{}manager}\PY{p}{,} \PY{n}{rc}
\PY{n}{font\PYZus{}path} \PY{o}{=} \PY{l+s+s2}{\PYZdq{}}\PY{l+s+s2}{C:/Windows/Fonts/NGULIM.TTF}\PY{l+s+s2}{\PYZdq{}}
\PY{n}{font} \PY{o}{=} \PY{n}{font\PYZus{}manager}\PY{o}{.}\PY{n}{FontProperties}\PY{p}{(}\PY{n}{fname}\PY{o}{=}\PY{n}{font\PYZus{}path}\PY{p}{)}\PY{o}{.}\PY{n}{get\PYZus{}name}\PY{p}{(}\PY{p}{)}
\PY{n}{rc}\PY{p}{(}\PY{l+s+s1}{\PYZsq{}}\PY{l+s+s1}{font}\PY{l+s+s1}{\PYZsq{}}\PY{p}{,} \PY{n}{family}\PY{o}{=}\PY{n}{font}\PY{p}{)}

\PY{k+kn}{import} \PY{n+nn}{cv2}
\PY{k+kn}{import} \PY{n+nn}{numpy} \PY{k}{as} \PY{n+nn}{np}
\PY{k+kn}{from} \PY{n+nn}{matplotlib} \PY{k+kn}{import} \PY{n}{pyplot} \PY{k}{as} \PY{n}{plt}

\PY{n}{img} \PY{o}{=} \PY{n}{cv2}\PY{o}{.}\PY{n}{imread}\PY{p}{(}\PY{l+s+s2}{\PYZdq{}}\PY{l+s+s2}{../../data/find\PYZus{}face.jpg}\PY{l+s+s2}{\PYZdq{}}\PY{p}{,}\PY{l+m+mi}{0}\PY{p}{)} \PY{c+c1}{\PYZsh{} gray scale로 읽어서, 1 채널 명암값에 필터를 적용한다.}

\PY{c+c1}{\PYZsh{} 가우시안 커널을 직접 생성}
\PY{n}{gaussian\PYZus{}k} \PY{o}{=} \PY{n}{gaussian}\PY{p}{(}\PY{l+m+mi}{6}\PY{p}{)}


\PY{n}{self\PYZus{}img} \PY{o}{=} \PY{n}{conv}\PY{p}{(}\PY{n}{src}\PY{o}{=}\PY{n}{img}\PY{p}{,}\PY{n}{kernel}\PY{o}{=}\PY{n}{gaussian\PYZus{}k}\PY{p}{)}
\PY{n}{gaussian\PYZus{}img} \PY{o}{=} \PY{n}{cv2}\PY{o}{.}\PY{n}{GaussianBlur}\PY{p}{(}\PY{n}{img}\PY{p}{,} \PY{p}{(}\PY{l+m+mi}{0}\PY{p}{,} \PY{l+m+mi}{0}\PY{p}{)}\PY{p}{,} \PY{l+m+mi}{6}\PY{p}{)}
\PY{n}{fiter\PYZus{}2d}\PY{o}{=}  \PY{n}{cv2}\PY{o}{.}\PY{n}{filter2D}\PY{p}{(}\PY{n}{img}\PY{o}{.}\PY{n}{astype}\PY{p}{(}\PY{n}{np}\PY{o}{.}\PY{n}{uint8}\PY{p}{)}\PY{p}{,}\PY{o}{\PYZhy{}}\PY{l+m+mi}{1}\PY{p}{,}\PY{n}{gaussian\PYZus{}k}\PY{p}{,}\PY{n}{borderType}\PY{o}{=}\PY{l+m+mi}{0}\PY{p}{)}
\PY{n+nb}{print}\PY{p}{(}\PY{l+s+s2}{\PYZdq{}}\PY{l+s+s2}{opencv gaussian 결과 }\PY{l+s+se}{\PYZbs{}n}\PY{l+s+s2}{\PYZdq{}}\PY{p}{,}\PY{n}{gaussian\PYZus{}img}\PY{p}{)}
\PY{n+nb}{print}\PY{p}{(}\PY{l+s+s2}{\PYZdq{}}\PY{l+s+s2}{filter 2d 사용시 }\PY{l+s+se}{\PYZbs{}n}\PY{l+s+s2}{\PYZdq{}}\PY{p}{,} \PY{n}{cv2}\PY{o}{.}\PY{n}{filter2D}\PY{p}{(}\PY{n}{img}\PY{o}{.}\PY{n}{astype}\PY{p}{(}\PY{n}{np}\PY{o}{.}\PY{n}{uint8}\PY{p}{)}\PY{p}{,}\PY{o}{\PYZhy{}}\PY{l+m+mi}{1}\PY{p}{,}\PY{n}{gaussian\PYZus{}k}\PY{p}{,}\PY{n}{borderType}\PY{o}{=}\PY{l+m+mi}{0}\PY{p}{)}\PY{p}{)}
\PY{n+nb}{print}\PY{p}{(}\PY{l+s+s2}{\PYZdq{}}\PY{l+s+s2}{self }\PY{l+s+se}{\PYZbs{}n}\PY{l+s+s2}{\PYZdq{}}\PY{p}{,} \PY{n}{self\PYZus{}img}\PY{p}{)}

\PY{n}{fig} \PY{o}{=}\PY{n}{plt}\PY{o}{.}\PY{n}{figure}\PY{p}{(}\PY{n}{figsize}\PY{o}{=}\PY{p}{(}\PY{l+m+mi}{10}\PY{p}{,}\PY{l+m+mi}{10}\PY{p}{)}\PY{p}{)}

\PY{n}{plt}\PY{o}{.}\PY{n}{subplot}\PY{p}{(}\PY{l+m+mi}{221}\PY{p}{)}\PY{c+c1}{\PYZsh{} 1행 2열중 1번째}
\PY{n}{plt}\PY{o}{.}\PY{n}{imshow}\PY{p}{(}\PY{n}{img}\PY{p}{,}\PY{l+s+s1}{\PYZsq{}}\PY{l+s+s1}{gray}\PY{l+s+s1}{\PYZsq{}}\PY{p}{)}
\PY{n}{plt}\PY{o}{.}\PY{n}{title}\PY{p}{(}\PY{l+s+s1}{\PYZsq{}}\PY{l+s+s1}{원본}\PY{l+s+s1}{\PYZsq{}}\PY{p}{)}

\PY{n}{plt}\PY{o}{.}\PY{n}{subplot}\PY{p}{(}\PY{l+m+mi}{222}\PY{p}{)}\PY{c+c1}{\PYZsh{} 1행 2열중 1번째}
\PY{n}{plt}\PY{o}{.}\PY{n}{imshow}\PY{p}{(}\PY{n}{gaussian\PYZus{}img}\PY{p}{,}\PY{l+s+s1}{\PYZsq{}}\PY{l+s+s1}{gray}\PY{l+s+s1}{\PYZsq{}}\PY{p}{)}
\PY{n}{plt}\PY{o}{.}\PY{n}{title}\PY{p}{(}\PY{l+s+s1}{\PYZsq{}}\PY{l+s+s1}{opencv gaussian}\PY{l+s+s1}{\PYZsq{}}\PY{p}{)}

\PY{n}{plt}\PY{o}{.}\PY{n}{subplot}\PY{p}{(}\PY{l+m+mi}{223}\PY{p}{)}\PY{c+c1}{\PYZsh{} 1행 2열중 1번째}
\PY{n}{plt}\PY{o}{.}\PY{n}{imshow}\PY{p}{(}\PY{n}{fiter\PYZus{}2d}\PY{p}{,}\PY{l+s+s1}{\PYZsq{}}\PY{l+s+s1}{gray}\PY{l+s+s1}{\PYZsq{}}\PY{p}{)}
\PY{n}{plt}\PY{o}{.}\PY{n}{title}\PY{p}{(}\PY{l+s+s1}{\PYZsq{}}\PY{l+s+s1}{filter2D}\PY{l+s+s1}{\PYZsq{}}\PY{p}{)}

\PY{n}{plt}\PY{o}{.}\PY{n}{subplot}\PY{p}{(}\PY{l+m+mi}{224}\PY{p}{)}\PY{c+c1}{\PYZsh{} 1행 2열중 1번째}
\PY{n}{plt}\PY{o}{.}\PY{n}{imshow}\PY{p}{(}\PY{n}{self\PYZus{}img}\PY{p}{,}\PY{l+s+s1}{\PYZsq{}}\PY{l+s+s1}{gray}\PY{l+s+s1}{\PYZsq{}}\PY{p}{)}
\PY{n}{plt}\PY{o}{.}\PY{n}{title}\PY{p}{(}\PY{l+s+s1}{\PYZsq{}}\PY{l+s+s1}{self}\PY{l+s+s1}{\PYZsq{}}\PY{p}{)}

\PY{c+c1}{\PYZsh{} 겉 테두리 문제는 padding 차이로 생각}
\PY{c+c1}{\PYZsh{} filter2D와 SELF는 반올림 차이.}
\end{Verbatim}
\end{tcolorbox}

    \begin{Verbatim}[commandchars=\\\{\}]
커널 사이즈:  37 37
305 458
18 18
입력 :
 [[0 0 0 {\ldots} 0 0 0]
 [0 0 0 {\ldots} 0 0 0]
 [0 0 0 {\ldots} 0 0 0]
 {\ldots}
 [0 0 0 {\ldots} 0 0 0]
 [0 0 0 {\ldots} 0 0 0]
 [0 0 0 {\ldots} 0 0 0]]
opencv gaussian 결과
 [[234 234 234 {\ldots} 253 253 253]
 [234 234 234 {\ldots} 253 253 253]
 [234 234 234 {\ldots} 253 253 253]
 {\ldots}
 [212 212 212 {\ldots} 238 238 238]
 [212 212 212 {\ldots} 238 238 238]
 [212 212 212 {\ldots} 238 238 237]]
filter 2d 사용시
 [[ 67  75  83 {\ldots}  89  81  72]
 [ 75  84  93 {\ldots} 100  91  81]
 [ 83  93 103 {\ldots} 111 100  89]
 {\ldots}
 [ 75  84  93 {\ldots} 104  94  84]
 [ 68  76  84 {\ldots}  94  85  76]
 [ 60  68  75 {\ldots}  84  76  68]]
self
 [[ 66.52560892  74.71636658  82.57050845 {\ldots}  89.44699959  80.93538508
   72.06031438]
 [ 74.72992721  83.9306287   92.75304663 {\ldots} 100.47655977  90.91514127
   80.94547081]
 [ 82.60169694  92.77126972 102.52251334 {\ldots} 111.05944094 100.49063804
   89.47061163]
 {\ldots}
 [ 74.63448132  83.84319123  92.67474237 {\ldots} 104.11746538  94.19632549
   83.85607522]
 [ 67.5362303   75.87284097  83.86950143 {\ldots}  94.17853956  85.20395026
   75.85036093]
 [ 60.13094249  67.55641607  74.68029475 {\ldots}  83.82499071  75.83653909
   67.51085678]]
    \end{Verbatim}

            \begin{tcolorbox}[breakable, size=fbox, boxrule=.5pt, pad at break*=1mm, opacityfill=0]
\prompt{Out}{outcolor}{112}{\boxspacing}
\begin{Verbatim}[commandchars=\\\{\}]
Text(0.5, 1.0, 'self')
\end{Verbatim}
\end{tcolorbox}
        
    \begin{center}
    \adjustimage{max size={0.9\linewidth}{0.9\paperheight}}{LogDog_files/LogDog_5_2.png}
    \end{center}
    { \hspace*{\fill} \\}
    
    \hypertarget{log-uxd544uxd130-laplacian-of-gaussian}{%
\subsection{- LOG 필터 (Laplacian of
Gaussian)}\label{log-uxd544uxd130-laplacian-of-gaussian}}

라플라시안은 이차편도함수를 더한것이다.

영상 f를 가우시안 스무딩하고, 다시 라플라시안을 구한다면 계산 효율이
떨어진다.

또한, 가우시안을 이산 필터로 근사화, 라플라시안을 이산 필터로 근사화
한다. 근사화가 두번되어 오차가 커진다.

그래서, LOG 필터를 사용한다.

컨볼루션과 라플라시안 연산 간 결합법칙이 성립한다.

그래서, 가우시안 커널에 라플라시안을 연산하고, 영상과 컨볼루션 한 것이
LOG 필터이다.

    \begin{tcolorbox}[breakable, size=fbox, boxrule=1pt, pad at break*=1mm,colback=cellbackground, colframe=cellborder]
\prompt{In}{incolor}{141}{\boxspacing}
\begin{Verbatim}[commandchars=\\\{\}]
\PY{c+c1}{\PYZsh{}ex) 라플라시안 가우시안 필터구현}
\PY{c+c1}{\PYZsh{}https://throwexception.tistory.com/1060}
\PY{k+kn}{import} \PY{n+nn}{cv2}
\PY{k+kn}{from} \PY{n+nn}{matplotlib} \PY{k+kn}{import} \PY{n}{pyplot} \PY{k}{as} \PY{n}{plt}

\PY{k}{def} \PY{n+nf}{laplacian\PYZus{}gaussian}\PY{p}{(}\PY{n}{sigma}\PY{p}{)}\PY{p}{:}
    \PY{c+c1}{\PYZsh{}커널 사이즈 할당}
    \PY{n}{kernel\PYZus{}size} \PY{o}{=} \PY{n}{np}\PY{o}{.}\PY{n}{around}\PY{p}{(}\PY{l+m+mi}{6}\PY{o}{*}\PY{n}{sigma}\PY{p}{)}
    \PY{k}{if} \PY{n}{kernel\PYZus{}size}\PY{o}{\PYZpc{}}\PY{k}{2} ==0:
        \PY{n}{kernel\PYZus{}size}\PY{o}{=}\PY{n}{kernel\PYZus{}size}\PY{o}{+}\PY{l+m+mi}{1}
    \PY{n+nb}{print}\PY{p}{(}\PY{l+s+s2}{\PYZdq{}}\PY{l+s+s2}{커널 사이즈: }\PY{l+s+s2}{\PYZdq{}}\PY{p}{,}\PY{n}{kernel\PYZus{}size}\PY{p}{,}\PY{n}{kernel\PYZus{}size}\PY{p}{)}
    \PY{n}{a} \PY{o}{=} \PY{n}{kernel\PYZus{}size}\PY{o}{/}\PY{o}{/}\PY{l+m+mi}{2}

    
    \PY{c+c1}{\PYZsh{}x,y좌표 생성}
    \PY{n}{x}\PY{p}{,} \PY{n}{y} \PY{o}{=} \PY{n}{np}\PY{o}{.}\PY{n}{ogrid}\PY{p}{[}\PY{o}{\PYZhy{}}\PY{n}{a}\PY{p}{:}\PY{n}{a}\PY{o}{+}\PY{l+m+mi}{1}\PY{p}{,} \PY{o}{\PYZhy{}}\PY{n}{a}\PY{p}{:}\PY{n}{a}\PY{o}{+}\PY{l+m+mi}{1}\PY{p}{]}
    \PY{c+c1}{\PYZsh{}print(x,y)}
    
    \PY{c+c1}{\PYZsh{}가우시안 식 대입.}
    \PY{n}{kernel} \PY{o}{=} \PY{p}{(}\PY{p}{(}\PY{n}{y}\PY{o}{*}\PY{o}{*}\PY{l+m+mi}{2}\PY{o}{+}\PY{n}{x}\PY{o}{*}\PY{o}{*}\PY{l+m+mi}{2}\PY{p}{)}\PY{o}{\PYZhy{}}\PY{l+m+mi}{2}\PY{o}{*}\PY{p}{(}\PY{n}{sigma}\PY{o}{*}\PY{o}{*}\PY{l+m+mi}{2}\PY{p}{)}\PY{p}{)}\PY{o}{*}\PY{n}{np}\PY{o}{.}\PY{n}{exp}\PY{p}{(}\PY{o}{\PYZhy{}}\PY{p}{(}\PY{n}{y}\PY{o}{*}\PY{o}{*}\PY{l+m+mi}{2}\PY{o}{+}\PY{n}{x}\PY{o}{*}\PY{o}{*}\PY{l+m+mi}{2}\PY{p}{)}\PY{o}{/}\PY{p}{(}\PY{l+m+mi}{2}\PY{o}{*}\PY{n}{sigma}\PY{o}{*}\PY{o}{*}\PY{l+m+mi}{2}\PY{p}{)}\PY{p}{)}\PY{c+c1}{\PYZsh{} 상수는 제외}
    \PY{c+c1}{\PYZsh{}라플라시안은 안나눠준다.}
    
    \PY{k}{return} \PY{n}{kernel}
\end{Verbatim}
\end{tcolorbox}

    \begin{tcolorbox}[breakable, size=fbox, boxrule=1pt, pad at break*=1mm,colback=cellbackground, colframe=cellborder]
\prompt{In}{incolor}{142}{\boxspacing}
\begin{Verbatim}[commandchars=\\\{\}]
\PY{n}{laplacian\PYZus{}gaussian}\PY{p}{(}\PY{l+m+mi}{1}\PY{o}{/}\PY{l+m+mi}{2}\PY{p}{)}
\PY{c+c1}{\PYZsh{}중앙이 \PYZhy{}로가는 모자모양}
\end{Verbatim}
\end{tcolorbox}

    \begin{Verbatim}[commandchars=\\\{\}]
커널 사이즈:  3.0 3.0
    \end{Verbatim}

            \begin{tcolorbox}[breakable, size=fbox, boxrule=.5pt, pad at break*=1mm, opacityfill=0]
\prompt{Out}{outcolor}{142}{\boxspacing}
\begin{Verbatim}[commandchars=\\\{\}]
array([[ 0.02747346,  0.06766764,  0.02747346],
       [ 0.06766764, -0.5       ,  0.06766764],
       [ 0.02747346,  0.06766764,  0.02747346]])
\end{Verbatim}
\end{tcolorbox}
        
    \begin{tcolorbox}[breakable, size=fbox, boxrule=1pt, pad at break*=1mm,colback=cellbackground, colframe=cellborder]
\prompt{In}{incolor}{152}{\boxspacing}
\begin{Verbatim}[commandchars=\\\{\}]
\PY{k+kn}{from} \PY{n+nn}{mpl\PYZus{}toolkits}\PY{n+nn}{.}\PY{n+nn}{mplot3d} \PY{k+kn}{import} \PY{n}{Axes3D}
\PY{n}{X} \PY{o}{=} \PY{n}{np}\PY{o}{.}\PY{n}{arange}\PY{p}{(}\PY{o}{\PYZhy{}}\PY{l+m+mi}{15}\PY{p}{,} \PY{l+m+mi}{16}\PY{p}{)}
\PY{n}{Y} \PY{o}{=} \PY{n}{np}\PY{o}{.}\PY{n}{arange}\PY{p}{(}\PY{o}{\PYZhy{}}\PY{l+m+mi}{15}\PY{p}{,} \PY{l+m+mi}{16}\PY{p}{)}
\PY{n}{z}\PY{o}{=}\PY{n}{laplacian\PYZus{}gaussian}\PY{p}{(}\PY{l+m+mi}{5}\PY{p}{)}
\PY{n}{XX}\PY{p}{,} \PY{n}{YY} \PY{o}{=} \PY{n}{np}\PY{o}{.}\PY{n}{meshgrid}\PY{p}{(}\PY{n}{X}\PY{p}{,} \PY{n}{Y}\PY{p}{)}
\PY{n}{fig} \PY{o}{=} \PY{n}{plt}\PY{o}{.}\PY{n}{figure}\PY{p}{(}\PY{p}{)}
\PY{n}{ax} \PY{o}{=} \PY{n}{Axes3D}\PY{p}{(}\PY{n}{fig}\PY{p}{)}
\PY{n}{ax}\PY{o}{.}\PY{n}{set\PYZus{}title}\PY{p}{(}\PY{l+s+s2}{\PYZdq{}}\PY{l+s+s2}{3D Surface Plot}\PY{l+s+s2}{\PYZdq{}}\PY{p}{)}
\PY{n}{ax}\PY{o}{.}\PY{n}{plot\PYZus{}surface}\PY{p}{(}\PY{n}{XX}\PY{p}{,} \PY{n}{YY}\PY{p}{,} \PY{n}{z}\PY{p}{,} \PY{n}{rstride}\PY{o}{=}\PY{l+m+mi}{1}\PY{p}{,} \PY{n}{cstride}\PY{o}{=}\PY{l+m+mi}{1}\PY{p}{,} \PY{n}{cmap}\PY{o}{=}\PY{l+s+s1}{\PYZsq{}}\PY{l+s+s1}{hot}\PY{l+s+s1}{\PYZsq{}}\PY{p}{)}
\PY{n}{plt}\PY{o}{.}\PY{n}{show}\PY{p}{(}\PY{p}{)}
\PY{c+c1}{\PYZsh{}시그마 5일때, 맥시칸햇.}
\end{Verbatim}
\end{tcolorbox}

    \begin{Verbatim}[commandchars=\\\{\}]
커널 사이즈:  31 31
    \end{Verbatim}

    \begin{Verbatim}[commandchars=\\\{\}]
c:\textbackslash{}users\textbackslash{}dlwhd\textbackslash{}miniconda3\textbackslash{}envs\textbackslash{}mlstudy\textbackslash{}lib\textbackslash{}site-
packages\textbackslash{}ipykernel\_launcher.py:7: MatplotlibDeprecationWarning: Axes3D(fig)
adding itself to the figure is deprecated since 3.4. Pass the keyword argument
auto\_add\_to\_figure=False and use fig.add\_axes(ax) to suppress this warning. The
default value of auto\_add\_to\_figure will change to False in mpl3.5 and True
values will no longer work in 3.6.  This is consistent with other Axes classes.
  import sys
    \end{Verbatim}

    \begin{center}
    \adjustimage{max size={0.9\linewidth}{0.9\paperheight}}{LogDog_files/LogDog_9_2.png}
    \end{center}
    { \hspace*{\fill} \\}
    

    % Add a bibliography block to the postdoc
    
    
    
\end{document}
